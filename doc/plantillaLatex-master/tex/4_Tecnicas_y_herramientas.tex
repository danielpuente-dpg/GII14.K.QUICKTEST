\capitulo{4}{Técnicas y herramientas}

En esta sección se llevará a cabo una mención y breve explicación sobre el conjunto técnicas y herramientas utilizadas durante el desarrollo del proyecto.

\section{Lenguajes}

\subsection{PHP}

PHP es un lenguaje de código abierto muy popular especialmente adecuado para el desarrollo web y que puede ser incrustado en HTML \cite{wiki:php} cuyas siglas es un acrónimo recursivo de PHP Hypertext Preprocesssor. Este lenguaje se utiliza principalmente en el lado del servidor y está enfocado en el desarrollo web para el contenido dinámico \cite{wiki:php2}.
Este lenguaje se ha decidido utilizar frente a otras alternativas en el lado del servidor debido principalmente a que el proyecto de partida se basa en este lenguaje para el acceso a la base de datos. Además, cabe destacar que desde mi punto de vista me parece una gran decisión al ser uno de los lenguajes más utilizados en el lado del servidor, existe una gran documentación al respecto y la curva de aprendizaje es menos costosa al asemejarse a otros lenguajes orientados a objetos.

\subsection{JSON}

JSON es un formato ligero de intercambio de datos cuyas siglas es un acrónimo de JavaScript Object Notation, es decir, Notación de Objetos de JavaScript \cite{wiki:json}. JSON es un formato de texto independientemente del lenguaje y se le considera una gran alternativa frente al formato de intercambio de datos XML al ser mucho más sencillo de parsear por un analizador sintáctico \cite{wiki:json2}.
Este lenguaje se ha decidido utilizar para la devolución de los diferentes datos a la aplicación Android. De esta forma, la comunicación de la aplicación se podrá realizar de una forma más sencilla con la base de datos. Además, las propias librerías de Android contienen ciertas clases para poder realizar de forma fácil un correcto tratamiento de los datos.

\subsection{Android}

\section{Herramientas}

\subsection{XAMPP}

XAMPP es un paquete de instalación independiente de la plataforma y de software libre. Dicho paquete contiene el sistema de gestión de bases de datos MySQL¸ el servidor web Apache y los interpretes para lenguajes de script: PHP y Perl \cite{wiki:xampp}. Su nombre proviene del acrónimo X indicando que es compatible para cualquier sistema operativo y el resto de siglas AMPP hace referencia a cada uno de los elementos que contiene el paquete, es decir, Apache, MySQL, PHP, Perl respectivamente.
Esta herramienta se ha decidido utilizar principalmente al ser utilizada en el proyecto de partida. Además, cabe destacar que me parece una gran decisión ya que el propio paquete contiene todo lo necesario para poder crear un servidor web.

\subsection{GitHub}

GitHub es una plataforma de desarrollo colaborativo para alojar proyectos utilizando un sistema de control de versiones Git \cite{wiki:github}. 
Se ha decidido optar por esta herramienta al haberla utilizado en ciertas ocasiones, al ser una de las más utilizadas para el desarrollo colaborativo y al existir un plugin ZenHub para la gestión desde el propio repositorio de las distintas tareas a realizar en un panel de trabajo. Además, cabe destacar que el repositorio permite dos tipos de formato para el repositorio: público y privado. En nuestro caso se encuentra público. \url{https://github.com/danielpuente-dpg/GII14.K.QUICKTEST}

\subsection{TortoiseSVN}

TortoiseSVN es un software libre utilizado para llevar a cabo el control de versiones del proyecto sobre GitHub \cite{wiki:tortoisesvn}. Implementa una extensión del shell de Windows, por lo que el control de versiones no tiene por qué realizarse mediante línea de comandos, sino que se puede realizar mediante la interfaz gráfica que proporciona esta herramienta la cual es muy fácil de utilizar.

\subsection{Moodle}
\subsection{PhpStorm}
\subsection{phpMyAdmin}
\subsection{PHPUnit}
\subsection{SonarQube}
\subsection{Advanced REST Client}
\subsection{StarUML}
\subsection{Librerías}

A continuación, se incluye una pequeña explicación sobre las librerías utilizadas para el desarrollo del proyecto.

\subsubsection{Consumir peticiones}

Durante el desarrollo de la aplicación se han barajado ambas librerías para en consumo de las peticiones al propio API y al webservice de Moodle. Finalmente, se ha decido optar por la librería Retrofit al ser más fácil de utilizar, más intuitiva a la hora de crear y configurar las distintas peticiones, más flexible y suele tener mejores resultados que Volley

\begin{itemize}

	\item \textbf{Retrofit:} Retrofit proporciona un framework para poder interactuar con APIs y enviar peticiones HTTP de forma fiable desde aplicaciones Android. Para ello proporciona un cliente HTTP encargado de interactuar con APIs y de gestionar las peticiones de manera transparente.
Esta librería es utilizada para el tratamiento de las peticiones desde la aplicación android. Además, permite consumir peticiones de manera síncrona y asíncrona. \cite{wiki:retrofit}
	
	\item \textbf{Volley:} Al igual que la librería anterior, Volley es una librería desarrollada por Google para enviar peticiones HTTP desde aplicaciones Android. \cite{wiki:volley}
	
	\item \textbf{Gson:} Esta librería proporcionada por Google, permite trabajar con JSON a la hora de serializar y deserializar los objetos. Es utilizada para convertir las respuestas JSON de las APIs en objetos Java de manera muy sencilla, eliminado toda esta carga de conversión al programador. Además, también permite el paso contrario a la hora de enviar objetos a las APIs, al encargarse de transformar estos objetos en respuestas en formato JSON. \cite{wiki:gson}

\end{itemize}

\section{Técnicas}

\subsection{LTI}
\subsection{MVC}
\subsection{SCRUM}

SCRUM es un modelo de referencia que define un conjunto de práctica y roles, que pueden utilizarse como punto de partida para llevar a cabo la definición de como se elaborará el proceso de desarrollo del proyecto \cite{wiki:scrum}. Cabe destacar que este modelo se encuentra dentro de los marcos de las metodologías ágiles y que es de los más utilizados actualmente.
Los roles principales son:

\begin{itemize}

	\item \textbf{Scrum Master:} se encarga de gestionar los cambios y procura facilitar la aplicación de esta metodología.
	
	\item \textbf{Product Owner:} en este grupo se encontrarán el personal interno o externo que representa al cliente. Este grupo se encargará de que el trabajo se realice de forma acorde con las necesidades y peticiones del cliente.
	
	\item \textbf{Team:} representa al equipo de desarrollo encargados de ejecutar el desarrollo y de entregar el producto deseado.
	
	\item \textbf{Product Owner:} en este grupo se encontrarán el personal interno o externo que representa al cliente. Este grupo se encargará de que el trabajo se realice de forma acorde con las necesidades y peticiones del cliente.	

\end{itemize}
\cite{wiki:scrum}
 
Para realizar el desarrollo del producto se deben de definir un Product Backlog, el cual contendrá todas las historias de usuario a realizar. Dichas historias de usuario se asignarán a los distintos Sprints o iteraciones que se realicen a lo largo del desarrollo del producto hasta finalizar con todas las historias de usuario. Se ha definido que la duración de los distintos Sprints sea de 2 semanas al ser lo más recomendable.
Además, cabe destacar que se ha decidido utilizar esta metodóloga de desarrollo principalmente al ser la metodología más recomendada a utilizar a lo largo de la carrera, tanto mis tutores como yo nos encontramos cómodos usando dicha metodología, se puede llevar a cabo ciertas entregas por cada sprint distribuyendo el trabajo de manera más adecuada y principalmente, garantizamos una mayor productividad y calidad del producto a entregar. 










