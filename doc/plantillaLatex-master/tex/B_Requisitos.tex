\apendice{Especificación de Requisitos}

\section{Introducción}

Como ya se ha comentado anteriormente, se va utilizar la metodología ágil SCRUM. Es por esto que no vamos a tratar las tareas a realizar como requisitos, sino como historias de usuario. Todas estas historias de usuarios formarán parte del Product Backlog y serán asignadas a determinadas iteraciones durante el desarrollo del proyecto.
A continuación, se incluye un listado sobre las historias de usuarios y sus correspondientes diagramas de casos de uso.


\section{Product Backlog}

En esta sección se incluyen todas aquellas historias de usuarios necesarias para los alumnos y los profesores.


\subsubsection{Historia de usuario 1: Iniciar sesión}

\tablaSmall{Historia de usuario 1}{l l}{coste-1}
{ \multicolumn{1}{l}{HU1}\\}{ 
Título & Como un usuario autenticado, podrá iniciar sesión. \\
Rol & Alumno o profesor autenticado. \\
Descripción & El usuario introducirá sus credenciales y se le mostrará \\\
& aquellos cursos en los que se encuentre matriculado. \\
Precondiciones & - El usuario debe estar previamente registrado. \\\
			   & - Haber iniciado sesión. \\
}

\subsubsection{Historia de usuario 7: Cerrar sesión}

\tablaSmall{Historia de usuario 7}{l l}{coste-1}
{ \multicolumn{1}{l}{HU7}\\}{ 
Título & Como un usuario autenticado, podrá cerrar sesión.\\
Rol & Alumno o profesor autenticado. \\
Descripción & Siempre y cuando el usuario se encuentre logeado, podrá cerrar sesión.\\
Precondiciones & - El usuario debe estar previamente registrado.\\\
& - Haber iniciado sesión. \\
}

\subsubsection{Historia de usuario 8: Recordar campos}

\tablaSmall{Historia de usuario 8}{l l}{coste-1}
{ \multicolumn{1}{l}{HU8}\\}{ 
Título & Como un usuario autenticado, podrá recordar los campos.\\
Rol & Alumno o profesor autenticado. \\
Descripción & El sistema permitirá recordar los campos con los que usuario inicia sesión.\\
Precondiciones & - El usuario debe estar previamente registrado.\\\
& - Haber iniciado sesión. \\
}

\subsubsection{Historia de usuario 12: Cursos}

\tablaSmall{Historia de usuario 12}{l l}{coste-1}
{ \multicolumn{1}{l}{HU12}\\}{ 
Título & Como un usuario autenticado, podrá ver sus cursos.\\
Rol & Alumno o profesor autenticado. \\
Descripción & El sistema mostrará los cursos en los que se encuentra matriculado.\\
Precondiciones & - El usuario debe estar previamente registrado.\\\
& - Haber iniciado sesión. \\
}

\subsection{Historias de usuario de los alumnos}

\subsubsection{Historia de usuario 2: Cuestionarios}

\tablaSmall{Historia de usuario 2}{l l}{coste-1}
{ \multicolumn{1}{l}{HU2}\\}{ 
Título & Como un usuario autenticado, el sistema notificará sobre los \\\
& cuestionarios a resolver. \\
Rol & Alumno autenticado. \\
Descripción & Una vez el usuario haya iniciado sesión obtendrá todos aquellos\\\  & cuestionarios en los que se encuentre asignado. \\
Precondiciones & - El usuario debe estar previamente registrado.\\\
& - Haber iniciado sesión. \\\
& - Existan cuestionarios en aquellos cursos en los que se encuentra\\\
& matriculado el alumno. \\
}

\subsubsection{Historia de usuario 3: Resolver cuestionario}

\tablaSmall{Historia de usuario 3}{l l}{coste-1}
{ \multicolumn{1}{l}{HU3}\\}{ 
Título & Como un usuario autenticado, podrá resolver un cuestionario. \\
Rol & Alumno autenticado. \\
Descripción & Una vez el usuario haya iniciado sesión podrá resolver aquellos\\\
& cuestionarios que se encuentren asignados, a un curso al que pertenezca. \\
Precondiciones & - El usuario debe estar previamente registrado.\\\
& - Haber iniciado sesión. \\\
& - Existan cuestionarios en aquellos cursos en los que se encuentra\\\
& registrado el alumno. \\
}

\subsubsection{Historia de usuario 4: Finalizar cuestionario}

\tablaSmall{Historia de usuario 4}{l l}{coste-1}
{ \multicolumn{1}{l}{HU4}\\}{ 
Título & Como un usuario autenticado, podrá finalizar un cuestionario. \\
Rol & Alumno autenticado. \\
Descripción & El usuario podrá finalizar aquel cuestionario que este resolviendo \\
Precondiciones & - El usuario debe estar previamente registrado.\\\
& - Haber iniciado sesión. \\\
& - Existan cuestionarios en aquellos cursos en los que se encuentra\\\
& registrado el alumno. \\\
& - Haber iniciado la resolución de un cuestionario. \\
}

\subsubsection{Historia de usuario 5: Calificación obtenida}

\tablaSmall{Historia de usuario 5}{l l}{coste-1}
{ \multicolumn{1}{l}{HU5}\\}{ 
Título & Como un usuario autenticado, el sistema notificará sobre la \\\
& calificación obtenida. \\
Rol & Alumno autenticado. \\
Descripción & Una vez el usuario finalice el cuestionario, el sistema \\\
& proporcionará una retroalimentación sobre el cuestionario resuelto. \\
Precondiciones & - El usuario debe estar previamente registrado.\\\
& - Haber iniciado sesión. \\\
& - Existan cuestionarios en aquellos cursos en los que se encuentra \\\
& registrado el alumno. \\\
& - Haber iniciado la resolución de un cuestionario. \\\
& - Haber finalizado un cuestionario. \\
}

\subsubsection{Historia de usuario 6: Revisar cuestionario resuelto}

\tablaSmall{Historia de usuario 6}{l l}{coste-1}
{ \multicolumn{1}{l}{HU6}\\}{ 
Título & Como un usuario autenticado, podrá revisar un cuestionario resuelto.\\\
Rol & Alumno autenticado. \\
Descripción & El usuario podrá revisar la retroalimentación de un cuestionario \\\ & que ya esté finalizado. \\
Precondiciones & - El usuario debe estar previamente registrado.\\\
& - Haber iniciado sesión. \\\
& - Existan cuestionarios en aquellos cursos en los que se encuentra \\\
& registrado el alumno. \\\
& - Haber iniciado la resolución de un cuestionario. \\\
& - Haber finalizado un cuestionario. \\
}

\subsection{Historias de usuario de los profesores}

\subsubsection{Historia de usuario 9: Cuestionarios}

\tablaSmall{Historia de usuario 9}{l l}{coste-1}
{ \multicolumn{1}{l}{HU9}\\}{ 
Título & Como un usuario autenticado, podrá ver los cuestionarios a los que \\\ & se enfrentan sus alumnos.\\
Rol & Profesor autenticado. \\
Descripción & El sistema mostrará al profesor los cuestionarios a los \\\
& que se enfrentan sus alumnos.\\
Precondiciones & - El usuario debe estar previamente registrado.\\\
& - Haber iniciado sesión. \\
}

\subsubsection{Historia de usuario 10: Calificaciones}

\tablaSmall{Historia de usuario 10}{l l}{coste-1}
{ \multicolumn{1}{l}{HU10}\\}{ 
Título & Como un usuario autenticado, podrá ver las calificaciones de los \\\ & alumnos en un cuestionario.\\
Rol & Profesor autenticado. \\
Descripción & El sistema mostrará al profesor la calificación de cada \\\ & alumno en cada cuestionario.\\
Precondiciones & - El usuario debe estar previamente registrado.\\\
& - Haber iniciado sesión. \\
}

\subsubsection{Historia de usuario 11: Ver cuestionarios}

\tablaSmall{Historia de usuario 11}{l l}{coste-1}
{ \multicolumn{1}{l}{HU11}\\}{ 
Título & Como un usuario autenticado, podrá ver los cuestionarios.\\
Rol & Profesor autenticado. \\
Descripción & El sistema mostrará al profesor los cuestionarios a los que se \\\ & enfrentan sus alumnos.\\
Precondiciones & - El usuario debe estar previamente registrado.\\\
& - Haber iniciado sesión. \\
}

\section{Diagrama de casos de uso}

En la siguiente ilustración podemos ver el diagrama de los casos de uso de nuestro sistema:

\imagen{DiagramaCasosDeUso}{Diagrama de casos de uso}

\subsection{Caso de uso 1: Iniciar sesión}

\tablaSmall{Caso de uso 1: Iniciar sesión}{l l}{coste-1}
{ \multicolumn{1}{l}{CU1}\\}{ 
Título & Iniciar sesión.\\
Descripción & El sistema permitirá al usuario autenticado iniciar sesión para \\\ & acceder al sistema, así los cuestionarios para ese usuario serán \\\
& visibles una vez el usuario inicie sesión.\\
Secuencia & 1. El usuario introducirá los campos de cuenta y contraseña. \\\
& 2. El sistema comprobará si los campos son correctos: \\\
& \hspace{0.25cm} 2.1. Si son correctos, el usuario pasa a estar logeado teniendo pleno \\\ & \hspace{0.25cm} acceso a las funcionalidades del sistema. \\\
& \hspace{0.25cm} 2.2. Si no son correctos, el sistema notificará que los campos son incorrectos. \\
Precondiciones & - El usuario debe estar previamente registrado en Moodle. \\
Comentarios & Este caso de uso, CU1, corresponde a la historia de usuario 1. \\	
}


\subsection{Caso de uso 2: Mostrar cuestionarios}

\tablaSmall{Caso de uso 2: Mostrar cuestionarios}{l l}{coste-1}
{ \multicolumn{1}{l}{CU2}\\}{ 
Título & Mostrar cuestionarios.\\
Descripción & El sistema permitirá al usuario ver los cuestionarios de QuickTest \\\
& en los que se encuentra \\
Secuencia & 1. El sistema mostrará los cuestionarios. \\
Precondiciones & - El usuario debe estar previamente registrado en Moodle. \\\ & - El usuario debe estar logeado en la aplicación para poder acceder \\\
& a esta funcionalidad. \\\
& - Deben existir cuestionarios de QuickTest en cursos en los que el usuario \\\ & se encuentre matriculado.\\
Comentarios & Este caso de uso, CU2, corresponde a la historia de usuario 2. \\
}

\subsection{Caso de uso 3: Mostrar cuestionarios resueltos}

\tablaSmall{Caso de uso 3: Mostrar cuestionarios resueltos}{l l}{coste-1}
{ \multicolumn{1}{l}{CU3}\\}{ 
Título & Mostrar cuestionarios resueltos.\\
Descripción & El sistema permitirá al usuario ver los cuestionarios de QuickTest \\\ & en los que se encuentra y ya ha resuelto. \\
Secuencia & 1. El sistema mostrará los cuestionarios resueltos. \\\
& \hspace{0.25cm} 1.1. Si existe algún cuestionario resuelto el usuario podrá seleccionar un  \\\ & \hspace{0.25cm} cuestionario y obtener la retroalimentación del mismo HU5, HU6. \\
Precondiciones & - El usuario debe estar previamente registrado en Moodle.\\\
& - El usuario debe estar logeado en la aplicación para poder acceder a esta \\\ & funcionalidad. \\\
& - Deben existir cuestionarios de QuickTest en cursos en los que el usuario  \\\ & se encuentre matriculado y que hayan sido resueltos.\\
Comentarios & Este caso de uso, CU3, corresponde a la historia de usuario 6. \\
}

\subsection{Caso de uso 4: Resolver cuestionario}

\tablaSmall{Caso de uso 4: Resolver cuestionario}{l l}{coste-1}
{ \multicolumn{1}{l}{CU4}\\}{ 
Título & Resolver cuestionario.\\
Descripción & El sistema permitirá al usuario resolver un cuestionario. \\
Secuencia & 1. El usuario seleccionará un cuestionario a resolver. \\\
& \hspace{0.25cm}1.1. Si el cuestionario está resuelto, el sistema mostrará la \\\
& \hspace{0.25cm} retroalimentación del mismo. \\\
& \hspace{0.25cm}1.2. Sino, el sistema mostrará el cuestionario a \hspace{0.25cm} resolver HU3. \\\
& \hspace{0.5cm} 1.2.1. Una vez finalizado el cuestionario, el sistema \hspace{0.25cm} enviará el \\\
& \hspace{0.5cm} cuestionario resuelto HU4. \\\
& \hspace{0.5cm} 1.2.2. El sistema mostrará la retroalimentación del cuestionario \\\ & \hspace{0.5cm} resuelto HU5, HU6. \\
Precondiciones & - El usuario debe estar previamente registrado en Moodle.\\\
& - El usuario debe estar logeado en la aplicación para poder acceder a esta \\\ & funcionalidad. \\\
& - Deben existir cuestionarios de QuickTest en cursos en los que el usuario  \\\ & se encuentre matriculado y que hayan sido resueltos.\\
Comentarios & Este caso de uso, CU4, corresponde a la historia de usuario 3, 4, 5, 6. \\
}

\subsection{Caso de uso 5: Olvidar campos}

\tablaSmall{Caso de uso 5: Olvidar campos}{l l}{coste-1}
{ \multicolumn{1}{l}{CU5}\\}{ 
Título & Recordar campos.\\
Descripción & El sistema permitirá al usuario olvidar los campos con las que \\\ & el usuario inicia sesión.\\
Secuencia & 1. El usuario seleccionara en el sistema cerrar sesión. \\\
& 2. El sistema olvidará las credenciales del usuario.\\
Precondiciones & - El usuario debe estar previamente registrado en Moodle.\\\
& - El usuario debe haber iniciado sesión y haber activado la opción 
\\\ & de recordar campos.\\
Comentarios & Este caso de uso, CU5, corresponde a la historia de usuario 7. \\
}

\subsection{Caso de uso 6: Recordar campos}

\tablaSmall{Caso de uso 6: Recordar campos}{l l}{coste-1}
{ \multicolumn{1}{l}{CU6}\\}{ 
Título & Olvidar campos.\\
Descripción & El sistema permitirá al usuario recordar los campos con las que \\\ & el usuario inicia sesión.\\
Secuencia & 1. El sistema recordará las credenciales del usuario. \\
Precondiciones & - El usuario debe estar previamente registrado en Moodle.\\
Comentarios & Este caso de uso, CU6, corresponde a la historia de usuario 8. \\
}





