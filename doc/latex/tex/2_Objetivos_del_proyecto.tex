\capitulo{2}{Objetivos del proyecto}

A continuación, se va a llevar a cabo una citación y argumentación de los distintos objetivos a realizar en este proyecto con el objetivo de transmitir la finalidad del mismo.

\section{Objetivos funcionales}

Partiendo del proyecto de partida identificamos los siguientes nuevos objetivos.

\subsection{Docentes}

\begin{itemize}

	\item Estos cuestionarios permitirán a los docentes poder llevar a cabo una evaluación de los conocimientos adquiridos por su alumnado.
	
	\item Los profesores podrán ver la calificación de sus alumnos desde su dispositivo.
	
	\item Los docentes podrán ver el cuestionario al que se enfrentan sus alumnos.
	
	\item Todas estas acciones podrán realizarse desde el propio dispositivo del profesor.	
	
	
\end{itemize}

\subsection{Alumnos}

\begin{itemize}

	\item Podrán resolver los distintos cuestionarios a los que tenga que enfrentarse desde su dispositivo.
	
	\item Podrán ver la calificación obtenida y revisar como han resuelto el cuestionario.
	
	\item Las calificaciones de los alumnos variarán en función de las distintas recompensas.
	
	\item Los alumnos serán evaluados en función de su orden de respuesta al cuestionario.
	
	\item Los alumnos podrán utilizar comodines si lo consideran necesarios y su calificación se vera afectada en caso de uso.
	

\end{itemize}

\section{Objetivos de carácter técnico}

\begin{itemize}
	
	\item Desarrollar una aplicación Android para la resolución de estos cuestionarios que sea compatible con el mayor numero de dispositivos.
	
	\item Desarrollar un API que se capaz de proporcionar una interfaz común para la comunicación del proyecto de partida con la aplicación a implementar.
	
	\item Utilizar un sistema de control de versiones como \emph{Git} que permita registrar los cambios realizados durante el desarrollo del proyecto y la utilización de una plataforma que permita alojar este tipo de proyectos como \emph{GitHub}.
		
	\item Desarrollar una aplicación Android que cumpla el diseño Material Design.
	
	\item Utilizar el sistema de compilación Gradle en Android para: \emph{automatizar, administrar el proceso de compilación del proyecto e incluir en el mismo, ciertas configuraciones de compilación personales}.
	
	\item Realizar el proyecto mediante la utilización de la metodología ágil: \emph{Scrum}.
	
	\item Utilización del web service de Moodle para obtener la información necesaria por la aplicación.

\end{itemize}


\section{Objetivos personales}

\begin{itemize}
	
	\item Enfrentarme a nuevos retos que pongan a prueba todos mis conocimientos adquiridos a lo largo del grado.
	
	\item Aprender a desarrollar aplicaciones Android.
	
	\item Enfrentarme a un posible proyecto o reto que se asemejen a los del mundo laboral.
	
	\item Construir y comprender un API.
	
	\item Enfrentarme a un proyecto utilizando una metodología ágil.
	\item Proporcionar una nueva forma de enfrentarnos a los cuestionarios del proyecto de partida.

\end{itemize}




















