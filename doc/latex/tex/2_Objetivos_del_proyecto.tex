\capitulo{2}{Objetivos del proyecto}

A continuación, se va a llevar a cabo una citación y argumentación de los distintos objetivos a realizar en este proyecto con el objetivo de transmitir la finalidad del mismo.

\section{Objetivos funcionales}

Partiendo del proyecto de partida identificamos los siguientes nuevos objetivos.

\subsection{Docentes}

\begin{itemize}

	\item Llevarán a cabo las tareas de crear, editar, publicar o duplicar un cuestionario en Moodle.
	\item Estos cuestionarios permitirán a los docentes poder llevar a cabo una evaluación de los conocimientos adquiridos por su alumnado.
	\item Al duplicar los cuestionarios estos podrán reutilizarse por los docentes para los distintos grupos o asignaturas que dicho docente imparta.
	\item Además, gracias a esta aplicación el docente podrá llevar a cabo todas estas funcionalidades con total comodidad desde su smartphone.
	
\end{itemize}

\subsection{Alumnos}

\begin{itemize}

	\item Ofrecer la posibilidad de resolver los distintos cuestionarios a los que tenga que enfrentarse desde su smartphone.
	\item Al finalizar el mismo, el sistema notificará al alumno de la calificación obtenido junto con una retroalimentación de las diferentes preguntas.
	\item Estas calificaciones podrán variar en función de distintas recompensas a la hora de enfrentarse al cuestionario.
	\item Estas recompensas o comodines a partir de ahora, permitirán al alumnado enfrentarse a la prueba de una manera más amigable al enmascarar la verdadera finalidad del cuestionario.

\end{itemize}

\section{Objetivos de carácter técnico}

La aplicación móvil deberá de poder ser lo suficientemente amigable para los distintos usuarios para facilitar su correcta utilización y finalidad.

La aplicación móvil a desarrollar se llevará a cabo para Android.

Para llevar a cabo la creación de la aplicación se utilizará el entorno de desarrollo Android Studio.

Versión: de la aplicación

Como conectaremos Android con la Web

API APLICACIÓN, FUNCIONE EN UN DISPOSITIVOS. VERSION.

\section{Objetivos personales}

Destacar que uno de los objetivos principales de este proyecto es adquirir nuevos conocimientos dentro del desarrollo Android, junto con otros conocimientos necesarios para poder llevar a cabo la correcta integración del proyecto de partida a el proyecto a desarrollar.

Enfrentarme a nuevos retos que pongan a prueba todos mis conocimientos adquiridos a lo largo del grado.

Enfrentarme a un posible trabajo o proyecto que se asemeje a mi vida profesional.



















