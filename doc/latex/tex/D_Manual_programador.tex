\apendice{Documentación técnica de programación}

\section{Introducción}

En esta sección se describen los requisitos mínimos para el funcionamiento de la aplicación, la configuración y las pruebas realizadas.

\section{Estructura de directorios}

La estructura es la siguiente:

\begin{itemize}
	\item \textbackslash{}: contiene el proyecto completo incluyendo la aplicación Android, la documentación, el proyecto de partida y el archivo README.
	\item \textbackslash{}doc\textbackslash{}latex\textbackslash{}: contiene la documentación del proyecto en formato \LaTeX.
	
	\item \textbackslash{}doc\textbackslash{}docs\textbackslash{}: contiene la documentación del proyecto en formato PDF.
	
	\item \textbackslash{}src\textbackslash{}\_QuickTest\_TFG\textbackslash{}: contiene el proyecto de partida junto con el APIREST creado, las pruebas de dicho API y las pruebas correspondientes al proyecto inicial.
	
	\item \textbackslash{}src\textbackslash{}\_QuickTest\_TFG\textbackslash{}app\textbackslash{}apiREST\textbackslash{}: contiene toda la lógica del APIREST.
	
	\item \textbackslash{}src\textbackslash{}\_QuickTest\_TFG\textbackslash{}app\textbackslash{}apiREST\textbackslash{}test\_apiRest\textbackslash{}: contiene las pruebas unitarias para comprobar el correcto funcionamiento del BackEnd.
	
	\item \textbackslash{}src\textbackslash{}InstalarBaseDeDatos\textbackslash{}: contiene la base de datos a importar.
	
	\item \textbackslash{}src\textbackslash{}QuickTest\_Android		\textbackslash{}: contiene toda la lógica del FrontEnd.
	
	\item \textbackslash{}src\textbackslash{}QuickTest\_Android		\textbackslash{}doc\textbackslash{}: contiene el javadoc de la aplicación.
	
	\item \textbackslash{}src\textbackslash{}QuickTest\_Android		\textbackslash{}app\textbackslash{}src\textbackslash{}main\textbackslash{}: contiene todas las clases, layouts y imagenes que forman la aplicación.
	
	\item \textbackslash{}src\textbackslash{}QuickTest\_Android		\textbackslash{}app\textbackslash{}src\textbackslash{}test\textbackslash{}: contiene las pruebas unitarias sobre el FrontEnd.
	
	\item \textbackslash{}src\textbackslash{}QuickTest\_Android		\textbackslash{}app\textbackslash{}src\textbackslash{}androidTest\textbackslash{}: contiene las pruebas automatizadas de UI sobre el FrontEnd.
	
	\item \textbackslash{}tst\textbackslash{}:contiene los test del proyecto de partida.
	
	
	
\end{itemize}

\section{Manual del programador}

\subsection{Entorno de desarrollo}\label{subsec:RequisitosMinimos}

Para el correcto funcionamiento del proyecto es necesario tener instalado las siguientes herramientas:

\subsubsection{Java JDK}

Java es un lenguaje de programación orientado a objetos y es el lenguaje mas popular para el desarrollo de aplicaciones Android. Para el desarrollo de este proyecto se ha utilizado la version 8. Esta versión se encuentra disponible en la web oficial de Java en el siguiente enlace: \url{http://www.oracle.com/technetwork/pt/java/javase/downloads/jdk8-downloads-2133151.html}

\subsubsection{Android Studio}

Android Studio es el entorno de desarrollo más utilizado para el desarrollo de aplicaciones Android. Además, es el IDE oficial recomendado por la plataforma. Cabe destacar que soporta la construcción basada en Gradle, incluye un propio emulador Android, refactorizaciones propias de Android, entre otras. Se encuentra disponible en el siguiente enlace: \url{https://developer.android.com/studio/index.html?hl=es-419}

\subsubsection{XAMPP}

XAMPP es un paquete de instalación de software libre utilizado como servidor de la aplicación. Este paquete contiene el servidor web Apache, base de datos MySQL, PHP y Perl. Se encuentra disponible en el siguiente enlace: \url{https://www.apachefriends.org/es/download.html}.

\subsubsection{Moodle}

Moodle es un software orientado a la educación, cuyo principal objetivo es el de facilitar un entorno de calidad para proporcionar cursos. Esta herramienta se encuentra disponible en el siguiente enlace: \url{https://download.moodle.org/}

\subsection{Instalación de las herramientas}\label{subsubsec:InstHerramientas}


\subsubsection{Servidor}

\begin{itemize}

	\item Acceder a la web oficial de XAMPP en el siguiente enlace:
\url{https://www.apachefriends.org/es/download.html}.

	\item Seleccionar la versión 5.6.30/PHP 5.6.30. Cabe destacar que es necesario instalar esta versión de XAMPP o inferiores, debido a que el proyecto de partida utiliza ciertas funcionalidades que se encuentran sin soporte en versiones superiores. Y comenzamos a descargar el instalador. \imagen{InstXammp}{Instalador de XAMMP}
	
	\item Una vez descargado, procedemos a la instalación de esta herramienta. A continuación, procedemos a realizar la típica instalación de Siguiente – Siguiente y seleccionamos donde ubicar los archivos a instalar.
	
	\item Al finalizar la instalación se lanzará una alerta del firewall de Windows indicando que se necesita conceder acceso a Apache. Marcamos todas las opciones y finalizamos. \imagen{AlertFirApache}{Alerta del firewall de Apache}
	\item Una vez instalado, ya tendremos instalado y disponible nuestro servidor XAMPP. Para el correcto funcionamiento del proyecto de partida y de QuickTest en Android es necesario arrancar Apache y MySQL al iniciar esta herramienta como podemos ver en la siguiente ilustración. \imagen{PanelXammp}{Panel de XAMPP}
	
	\item Además, al arrancar estos servicios nos volverá a saltar el firewall de Windows pidiendo conceder permisos a MySQL. \imagen{AlertFirMysql}{Alerta del firewall de MySQL}
	
	\item Concedemos los permisos y la instalación de nuestro servidor a finalizado.


\end{itemize}

\subsubsection{Moodle}\label{subsec:moodle}

Respecto al proyecto de partida se decidió utilizar como LMS, el mismo utilizado por la Universidad de Burgos, Moodle. Al tener que basarse en este proyecto no se ha realizado ningún cambio en la herramienta utilizada como LMS.
No obstante, en esta ocasión la versión de Moodle empleada es la más actual hasta la fecha Moodle 3.2.2+. Este cambio en la versión utilizada es realizado al tener que utilizar las funciones que proporciona el propio web services de Moodle. De manera que cuanto más actual sea la versión, más funcionalidades tendremos a nuestra disposición.

\begin{itemize}

	\item Descargamos el archivo comprimido de Moodle correspondiente a la versión 3.2.2+: \url{https://sourceforge.net/projects/moodle/files/Moodle/stable32/moodle-3.2.2.zip/download}
	
	\item Seleccionamos la versión comentada anteriormente y comenzamos la descarga del archivo comprimido. \imagen{FolderMoodle}{Carpeta comprimida de Moodle}
	
	\item Una vez finalizada la descarga, descomprimimos el archivo en la carpeta htdocs. Esta carpeta se encuentra en el directorio raíz de XAMPP. Si la ruta de instalación de XAMPP se ha realizado con los campos por defecto, esta se encontrará en ''C:\textbackslash{xampp}\textbackslash{htdocs}''. Esta es la carpeta raíz del localhost, desde donde el servidor XAMPP comenzará a buscar cuando le mandemos una petición.
	
	\item Una vez descomprimido la carpeta abrimos un navegador y realizamos una petición al servidor indicando que queremos acceder a la URL de Moodle. \imagen{urlMoodle}{url de Moodle}
	
	\item Al no tener instalada ni configurada la herramienta, nos pedirá seleccionar el idioma deseado. Una vez seleccionado el idioma, tendremos que seleccionar la ruta de instalación de los nuevos componentes. En este caso dejamos la configuración por defecto como podemos ver en la siguiente ilustración. \imagen{UbicInstDir}{Ubicación de donde instalar los directorios}
	
	\item Posteriormente, tendremos que seleccionar el tipo de base de datos a utilizar. En este caso la opción elegida es MariaDB para evitar problemas de compatibilidad. \imagen{SelBD}{Selección de la base de datos}
	
	\item Al finalizar la selección de la base de datos la instalación nos permitirá modificar la configuración por defecto de la misma. En este caso solamente añadiremos root en la sección Usuario de la base de datos como podemos ver en la siguiente ilustración y aceptamos la licencia. \imagen{ConfMoodle}{Configuración de Moodle}
	
	\item Antes de pulsar en el botón de siguiente tendremos que abrir una nueva pestaña en nuestro navegador e introducir el siguiente enlace localhost/phpmyadmin. \imagen{urlMysql}{url de MySql}
	
	\item Una vez nos encontremos en la página, tendremos que cambiar el cotejamiento de la conexión al servidor como podemos ver en la siguiente ilustración.\imagen{CambForBD}{Cambiamos el formato de la base de datos}
	
	\item En esa misma página seleccionamos en Base de datos. Dentro de esta sección en el apartado Crear base de datos introducimos la siguiente información y la creamos. \imagen{CrecTabMoodle}{Creación de la tabla Moodle con el formato deseado}
	
	\item Una vez realizada la creación de la base de datos de manera manual, podemos volver a la pestaña de instalación de Moodle y proseguir con la instalación.

	\item Si hemos realizado los anteriores pasos de forma correcta el sistema nos comunicará que el entorno cumple todos los requerimientos mínimos y continuamos con la instalación. Este proceso puede durar varios minutos dependiendo de la conexión a Internet.
	
	\item Una vez realizada la instalación de Moodle, es necesario la configuración de la plataforma. Para ello es necesario establecer un administrador de la misma. En nuestro caso la configuración es la que se puede ver en la siguiente ilustración. \imagen{ConfDatAdminMoodle}{Configuramos los datos del administrador de Moodle}
	
	\item A continuación, tendremos que configurar el sitio. Para ello introducimos la siguiente información. \imagen{ConfSites}{Configuramos el sitio}
	
	\item Una vez realizamos de forma correcta todos los anteriores pasos, la instalación y configuración de Moodle ha finalizado

\end{itemize}

\subsubsection{Configuración del Webservice Moodle}

Antes de comenzar a la explicación de los distintos pasos a realizar, tendremos que encontrarnos logeado con los credenciales del administrador del sitio.

\begin{itemize}

	\item Una vez logeado nos dirigimos a Administración del sitio. Allí seleccionamos la categoría Extensiones y dentro de esta categoría en Servicios Web.
	\item Si hemos realizado el punto anterior de manera correcta, nos encontramos en panel central de la configuración del Web Service. Este entorno de configuración se puede observar en la siguiente ilustración.
	\imagen{PanelConfWSM}{Panel de configuración del Web Service de Moodle}
	
	\item Para llevar a cabo la correcta configuración vamos a realizar los pasos de manera ordenada de menor a mayor como se puede ver en la anterior ilustración. 
	\begin{itemize}
	
		\item \textbf{Habilitar los Servicios Web:} para ello entramos en el paso 1, y activamos el checkbox y guardamos los cambios.
		
		\item \textbf{Administrar protocolos:} para poder interactuar con el Web Service tendremos que habilitar al menos uno de ellos. En nuestro caso utilizaremos REST al ser el mas común utilizado en la comunidad de los Web Service de Moodle. Además, permite mostrar el contenido de retorno en formato JSON, el cual es muy ventajoso para el el tratamiento de los datos en la aplicación Android. Por último, es aconsejable habilitar la documentación al ser un gran material de apoyo y guardamos los cambios.
		
		\item Como se ha comentado anteriormente, vamos a realizar la configuración de manera ordenada. No obstante, en el este ejemplo vamos a ignorar los pasos 3, 4, 5 y 6. En estos pasos, se crea un usuario especifico para interactuar con el Web Service, este usuario debe de tener ciertos privilegios para poder desempeñar esta funcionalidad, se crea un servicio y se agregan las funcionalidades de interés a dicho servicio. Estos pasos son ignorados, ya que se va ha utilizar el propio usuario administrador del sitio, al tener todos los privilegios necesarios y de esta manera no limitaremos las funcionalidades del Web Service al incluir todas ellas. Destacar, que no es recomendable utilizar el mismo usuario administrador de Moodle por motivos de seguridad, para interactuar con el Web Service. Al ser un trabajo meramente didáctico, utilizaremos este mismo usuario para evitar complicaciones.
Por último, estos pasos omitidos se encuentran explicados en el siguiente enlace:
\url{http://www.vitalinnova.com/servicios-web-en-moodle-2-0/}
		
		\item \textbf{Seleccione un servicio:} para poder habilitar el servicio externo tendremos que seleccionar el paso 5. Una vez allí, click sobre editar en servicio \emph{Moodle mobile web service} y activamos el checkbox.
		
	
	\end{itemize}
	
La configuración del Web Service ha finalizado. Para poder comprobar el correcto funcionamiento del mismo, vamos a pedir a este, el token del administrador. Para ello abrimos un nuevo navegador e introducimos la siguiente url: \url{http://localhost/moodle/login/token.php?username=admin&password=Asdf1234!&service=moodle_mobile_app}. Si todo ha ido correctamente la respuesta será la siguiente:
\imagen{ResTokAdmin}{Obtención del token del administador}
	
\end{itemize}

\subsubsection{Instalar base de datos}

A continuación, vamos a importar la base de datos que utiliza QuickTest. Esta base de datos es la misma proporcionada por el proyecto de partida, pero se ha incluido una nueva tabla para el almacenamiento de las notas de los cuestionarios resueltos desde la aplicación móvil. Para llevar a cabo esta tarea tendremos que seguir los siguientes pasos:

\begin{itemize}

	\item Dentro del proyecto nos situamos en la ruta src/InstalarBaseDeDatos.
	
	\item En esta carpeta tendremos un único archivo llamado: instalar\_BaseDatosQuickTest.sql
	
	\item Abrimos una nueva pestaña en nuestro navegador e introducir el siguiente enlace. \imagen{urlMysql}{Url de Mysql}
	
	\item Una vez nos encontremos en el panel de administración de la base de datos tendremos que importar el archivo del punto 2. Para ello seguimos los siguientes pasos. \imagen{FichBD}{Añadimos el fichero de la base de datos}
	
	\item Una vez importado el archivo seleccionamos continuar y el proceso de importación de la base de datos de QuickTest ha finalizado.

\end{itemize}

\subsubsection{Instalar/Configurar QuickTest}

Para instalar QuickTest tendremos copiar la carpeta \_QuickTest\_TFG en la carpeta raíz del localhost de XAMPP para que el servidor pueda buscar el directorio pedido al mandarle una petición.
Para ello tendremos que:

\begin{itemize}

	\item Copiar la carpeta \_QuickTest\_TFG en la carpeta htdocs de XAMPP. \imagen{FolderHtdocs}{Añadimos la carpeta al htdocs}

\end{itemize}

\subsection{Registrar un usuario en QuickTest}

Para poder utilizar QuickTest desde un LMS, en este caso Moodle, tendremos que estar previamente registrado en QuickTest. Para ello tendremos que:

\begin{itemize}

	\item Abrir una nueva pestaña en el navegador e introducir el siguiente enlace: \url{http://localhost/_QuickTest_TFG/app/views/managementView/startQuickTest_View.php}
	\imagen{UrlIniQT}{Abrimos la página de inicio de QuickTest}
	
	\item Seleccionamos Acceder/Registrarse. Una vez seleccionado este botón aparecerá una ventana emergente que permitirá Entrar o Registrarse.
	
	\item Para llevar a cabo el registro de un nuevo usuario en QuickTest tendremos que seguir dos criterios importantes.
	\begin{itemize}
         \item En el campo Email tendremos que usar el mismo que en Moodle.
         \item En el campo Contraseña podremos utilizar cualquiera siempre y cuando la contraseña este formada entre 5 y 10 caracteres.
    \end{itemize}
    En la siguiente ilustración podemos observar un inicio de sesión. Este usuario se encuentra por defecto en la base de datos inicial por lo que si se desea podemos saltarnos el paso de registrar un usuario. La contraseña es 12345. \imagenAncho{EntQT}{Entramos en QuickTest}{0.5}
    \item Una vez realizado el registro y su posterior inicio de sesión el sistema informará con una ventana de bienvenida. \imagenAncho{PanFF}{Panel de primeros pasos}{0.5}
    
    \item Al pinchar sobre Publicar cuestionario se abrirá una ventana emergente. Es, en esta ventana, donde se encuentra los tres campos necesarios para poder publicar y enlazar nuestros cuestionarios con Moodle. La información necesarios es la siguiente. \imagen{WPubCues}{Ventana emergente de publicar un cuestionario}
    
	
\end{itemize}


\subsection{Configurar/Publicar un cuestionario en Moodle}

En esta sección vamos a configurar Moodle para poder utilizar QuickTest. Para llevar a cabo esta tarea tendremos que seguir los siguientes pasos.

\subsubsection{Configuración}


\begin{itemize}

	\item Abrimos una nueva pestaña del navegador y accedemos a Moodle.
	
	\item Una vez allí tendremos que iniciar sesión con el nombre de usuario y contraseña propios del administrador del sistema, en este caso son las credenciales utilizadas en la sección \ref{subsec:moodle}. \imagen{InitSesMoodle}{Inicio de sesión de Moodle}
	
	\item Una vez iniciada la sesión tendremos que crear un curso donde añadir nuestro cuestionario, para ello tendremos que:
	
	\begin{itemize}
	
		\item Entramos en Administración del sitio.
		
		\item Una vez allí entramos en la categoría de Cursos y dentro de esa sección en Administrar cursos y categorías. \imagen{CrearCurso}{Creamos un curso}
		
		\item Una vez en esa sección seleccionamos Crear un nuevo curso y rellenamos los campos con la siguiente información. Al finalizar guardamos la información.
		\imagen{RelCampos}{Rellenamos los campos del nuevo curso}
	\end{itemize}
	
	\item Al crear el curso tendremos que matricular a los usuarios. Para ello vamos a crear dos usuarios.
	
	\begin{itemize}
		
		\item Tendremos que entrar en Administración del sitio, una vez allí tendremos que ir a la categoría de Usuarios y dentro de allí a la sección Crear un nuevo usuario. \imagen{CreNueUsers}{Creamos los nuevos usuarios}
		
		\item A continuación, tendremos que crear dos usuarios como podemos ver en las siguientes ilustraciones. \imagen{RelCmps}{Rellenamos los campos del primer usuario} \imagen{RelCmps2}{Rellenamos los campos del segundo usuario}
	\end{itemize}
	
	\item Una vez creados los usuarios, tendremos que matricular dichos usuarios en el curso creado para ello tendremos que entrar en Administración del sitio, una vez allí tendremos que ir a la categoría de Cursos y dentro de allí a la sección Administrar cursos y categorías.
	
	\item Allí seleccionamos el nuevo curso creado y pinchamos en Usuarios matriculados.
	
	\item A continuación, matriculamos ambos usuarios creados anteriormente, el alumno con rol de Estudiante y el profesor con el rol de Profesor.
	
	

\end{itemize}

\subsubsection{Publicar}

Para llevar a cabo la publicación de un cuestionario tendremos que:

\begin{itemize}

	\item Tendremos que iniciar sesión con algún usuario con rol de profesor.
	
	\item Una vez allí, seleccionamos un curso, activamos la edición y añadimos una herramienta externa.
	
	\begin{itemize}
	
			\item Para ello, tendremos que seleccionar Agregue una actividad o recurso. \imagen{AgHE}{Agregamos una nueva herramienta externa}
			\item En este instante comenzamos a configurar la herramienta externa para poder utilizar QuickTest. Tendremos que rellenar los campos con la siguiente información.\imagen{ConfHE2}{Configuramos la nueva herramienta externa 1} \imagen{ConfHE}{Configuramos la nueva herramienta externa 2}
			
			\item Además, revisar que en el apartado Privacidad se encuentren activas las siguientes opciones. \imagen{CamPrivi}{Campos de la privacidad}
	
	\end{itemize}
	
	\item En este instante la publicación ha finalizado.

\end{itemize}

\section{Pruebas del sistema}

Para comprobar el correcto funcionamiento se han desarrollado un conjunto de pruebas.

\subsection{APIREST}
En esta sección se recogen el conjunto de pruebas utilizadas para comprobar el correcto funcionamiento del BackEnd de la aplicación. Destacar que solamente se han realizado pruebas unitarias sobre el APIREST, ya que el único cometido de este es el de obtener y añadir información.

\subsubsection{Pruebas unitarias}

Las pruebas unitarias comprueban el funcionamiento de una única función, independientemente del resto. Para la construcción de este tipo de pruebas se ha utilizado PHPUnit. Este framework se instala al instalar XAMPP.

\subsubsection{Ejecución de las pruebas unitarias}

Las pruebas unitarias se encuentran dentro del propio paquete que contiene al APIREST añadido al proyecto de partida dentro de la carpeta apiREST.test\_apiRest. 

\imagenAncho{test-apiRest}{Contenido de la carpeta test\_apiRest}{0.5}

Dentro de este se encuentran las cuatro pruebas creadas para comprobar el correcto funcionamiento de cada una de las clases. Para llevar cabo la ejecución de dichas pruebas, podemos ejecutarlas de manera individual utilizando el respectivo archivo .bat de la clase  o ejecutar todos mediante el archivo runAll.bat

\subsection{Android}

En esta sección se recogen el conjunto de pruebas utilizadas para comprobar el correcto funcionamiento del FrontEnd de la aplicación.

\subsubsection{Pruebas unitarias}

Como ya hemos comentado anteriormente, las pruebas unitarias comprueban el correcto funcionamiento de un método o funcionalidad independientemente del resto. Es por esto, que se ha decidido realizar pruebas sobre todas aquellas estructuras encargadas de contener la información proporciona por el web services de Moodle, el propio APIREST o las propias estructuras de la aplicación.

\imagenAncho{pruebasunitariasAndroid}{Conjunto de pruebas unitarias del FrontEnd.}{0.5}

Estas pruebas podrán ejecutarse de dos maneras distintas:

\begin{itemize}
	\item \emph{Desde la propia consola de Windows:}
		\begin{list}{-}{}
			\item Situarnos en la raíz del proyecto QuickTest\_Android.
			\item Ejecutar: gradlew.bat testReleaseUnitTest
			
			\item Para visualizar los resultados abrir el archivo index.html en: QuickTest\_Android.app.build.reports.tests.testReleaseUnitTest
		\end{list}
	\item \emph{Desde Android Studio:} para ello tendremos que utilizar Android Studio, situarnos en la carpeta en la se encuentra dichas pruebas, botón derecho y Run Test.
	
\end{itemize}


\imagen{pruebasunitarias-android-salida}{Ejecución y resultado de las pruebas unitarias sobre el FrontEnd desde Android Studio.}

\subsubsection{Pruebas automatizadas de UI}

Este tipo de pruebas consiste en crear una batería de pruebas automatizada, que comprueben el correcto funcionamiento de la interfaz de usuario y valide los controles de la misma. 

\imagenAncho{pruebasui}{Conjunto de interfaz de usuario del FrontEnd.}{0.5}

Como podemos observar se ha decidido crear un caso de prueba por caso de uso. Al necesitar un emulador o un dispositivo conectado para poder ejecutar estas pruebas, tendremos que utilizar Android Studio. Para ello,  nos situamos en la carpeta en la que se encuentran dichas pruebas, botón derecho sobre la prueba a ejecutar y Run Test.

\subsubsection{Pruebas de estrés}

Las pruebas de estrés se utilizan para comprobar el funcionamiento de la aplicación bajo grandes cargas de trabajo. Es por esto, que se ha decidido realizar estas pruebas en la aplicación para localizar posibles perdidas de información.

Para ello se ha utilizado la herramienta Monkey de Android. Esta herramienta lanza en un emulador o dispositivo, un conjunto de flujos o eventos pseudo-aleatorias sobre la propia interfaz de la aplicación.

Para llevar a cabo la ejecución de esta prueba tendremos que:

\begin{enumerate}
		\item Tener iniciado el emulador de Android Studio.
	
	\item Ejecutar el archivo testMonkey.bat, situado en la raiz del proyecto Android.

\end{enumerate}

\subsubsection{Pruebas de cobertura}

Las pruebas de cobertura son las encargadas de medir la cantidad de código de un programa que nuestras pruebas cubren. En nuestro caso al realizar únicamente pruebas sobre las estructuras de almacenamiento de datos el nivel de código cubierto será pequeño.
Para realizar este tipo de pruebas se ha utilizado el propio entorno de Android Studio.

\begin{enumerate}

	\item Añadir al build.gradle de la aplicación. \imagenAncho{pruebascoberturaGradle}{Habilitamos las pruebas de cobertura.}{0.5}
	De esta manera permitimos la ejecución de este tipo de pruebas desde el propio IDE.
	\item Nos situamos en la carpeta que contiene los fuentes, botón derecho, Run Test Coverage.
	\imagenAncho{pruebascobertura}{Ejecución de las pruebas de cobertura sobre el paquete models.}{0.5}


\end{enumerate}


\subsubsection{Pruebas de integración continua}

Las pruebas de integración continua permite durante el desarrollo de una aplicación detectar posibles fallos cuantos antes. Para ello se ha decidido utilizar la herramienta sonarQube en su distribución online.
Para ello es necesario incluir en el proyecto:

\begin{enumerate}
	\item Seguir las indicaciones del vídeo proporcionado por el tribunal para la creación de una cuenta, generar el token, etc.
	\item Dentro del proyecto tendremos que añadir:
		\begin{enumerate}
			\item Añadir al build.gradle del proyecto. \imagenAncho{DependenciaSonarQube}{Añadimos la dependencia de sonarQube.}{0.5}
			\item Añadir al gradle.properties los parametros necesarios para su funcionamiento.\imagenAncho{confSonarQ}{Configuración de SonarQube.}{0.8}
		\end{enumerate}
	\item Para ejecutar esta prueba dentro de Android Studio bastará con introducir el siguiente comando. \imagenAncho{EjecSQ}{Ejecución de SonarQube}{0.5}
	
	\item El resultado de la prueba podrá visualizarse dentro de la distribución online de la herramienta.
\end{enumerate}



