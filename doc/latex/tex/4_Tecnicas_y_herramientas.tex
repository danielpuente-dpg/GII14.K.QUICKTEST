\capitulo{4}{Técnicas y herramientas}

En esta sección se llevará a cabo una mención y breve explicación sobre el conjunto técnicas y herramientas utilizadas durante el desarrollo del proyecto.

\section{Lenguajes}

\subsection{PHP}

PHP es un lenguaje de código abierto muy popular especialmente adecuado para el desarrollo web y que puede ser incrustado en HTML \cite{wiki:php} cuyas siglas es un acrónimo recursivo de PHP Hypertext Preprocesssor. Este lenguaje se utiliza en el lado del servidor y está enfocado en el desarrollo web para el contenido dinámico \cite{wiki:php2}.

Este lenguaje se ha decidido utilizar frente a otras alternativas en el lado del servidor, debido a que, el proyecto de partida se basa en este lenguaje en el lado del servidor. Además, cabe destacar que, desde mi punto de vista me parece una gran decisión al ser uno de los lenguajes más utilizados en el lado del servidor, existe una gran documentación al respecto y la curva de aprendizaje es menos costosa al asemejarse a otros lenguajes orientados a objetos.

\subsection{JAVA}

Java es un lenguaje de programación orientado a objetos, diseñado para disminuir el número de dependencias \cite{wiki:java}.

Además, es el lenguaje más utilizado para el desarrollo de aplicaciones Android y el lenguaje oficial recomendado. Por todo esto, se ha decidido utilizar este lenguaje para la creación de la aplicación para dispositivos Android.


\section{Herramientas}

\subsection{XAMPP}

XAMPP es un paquete de instalación independiente de la plataforma y de software libre. Dicho paquete contiene el sistema de gestión de bases de datos MySQL¸ el servidor web Apache y los interpretes para lenguajes de script: PHP y Perl \cite{wiki:xampp}. Su nombre proviene del acrónimo X, indicando que es \textbf{compatible para cualquier sistema operativo} y el resto de siglas AMPP, hace referencia a cada uno de los elementos que contiene el paquete, \textbf{Apache, MySQL, PHP, Perl} respectivamente.

Se ha decidido utilizar esta herramienta como servidor, al ser utilizada en el proyecto de partida. Además, cabe destacar, que me parece una gran decisión ya que el propio paquete contiene todo lo necesario para poder crear un servidor web.

\subsection{GitHub}

GitHub es una plataforma de desarrollo colaborativo para alojar proyectos utilizando un sistema de control de versiones Git \cite{wiki:github}.

Se ha decidido optar por esta herramienta, al haber sido anteriormente, al ser una de las más utilizadas para el desarrollo colaborativo y al existir el plugin ZenHub para la gestión desde el propio repositorio de las distintas tareas a realizar en un panel de trabajo.
 
Además, cabe destacar que el repositorio permite dos tipos de formato para el repositorio: público y privado. En nuestro caso se encuentra público. \url{https://github.com/danielpuente-dpg/GII14.K.QUICKTEST}

\subsection{ZenHub}\label{zenhub}

ZenHub es la herramienta utilizada para la gestión del proyecto.  Esta herramienta, permite su integración en la gestión de proyectos en GitHub, proporciona un tablero kanban para gestión de las tareas y automatiza la creación de los gráficos brundown para cada iteración. Gracias a este plugin gratuito, se facilita la gestión del proyecto de manera ágil. \cite{wiki:zenhub}

\subsection{TortoiseSVN}

TortoiseSVN es un software libre utilizado para llevar a cabo el control de versiones del proyecto sobre GitHub \cite{wiki:tortoisesvn}. Implementa una extensión del shell de Windows, por lo que el control de versiones no tiene por qué realizarse mediante línea de comandos, sino que se puede realizar mediante la interfaz gráfica que proporciona esta herramienta, la cual es muy fácil de utilizar.

\subsection{JSON}\label{json}

JSON es un formato ligero de intercambio de datos cuyas siglas es un acrónimo de JavaScript Object Notation, es decir, Notación de Objetos de JavaScript \cite{wiki:json}. 

JSON es un formato de texto independientemente del lenguaje y se le considera una gran alternativa frente al formato de intercambio de datos XML, al ser mucho más sencillo de parsear por un analizador sintáctico \cite{wiki:json2}.

Se ha decidido utilizar para la captación y devolución de los diferentes datos al proyecto de partida y a la aplicación Android respectivamente. De esta forma, la comunicación de la aplicación
con el proyecto se simplifica, gracias a la utilización de ciertas librerías que facilitan el tratamiento de los datos en Android, como es \emph{Gson}.

\subsection{Android}

Para llevar a cabo el desarrollo se decidió utiliza el propio entorno de desarrollo oficial y recomendado por Google: \emph{Android Studio.}

Android Studio, es un IDE que soporta la construcción basada en Gradle, incluye su propio emulador Android y refactorizaciones propias, entre otras características.

\subsection{Moodle}

Moodle es un software orientado a la educación, cuyo principal objetivo es el de facilitar un entorno de calidad para proporcionar cursos. \cite{wiki:moodle}

Se ha decidido utilizar este software al ser utilizado por el proyecto de partida como consumidor del proyecto.


\subsection{PhpStorm}

PhpStorm es el entorno de desarrollo comercial para PHP, HTML y JavaScript de JetBarins IntelliJ IDEA utilizado para el desarrollo del API. \cite{wiki:phpStorm}

Se ha decidido utilizar este IDE frente a otras alternativas como \textbf{Notepad++} \cite{wiki:notepad++} al ser un entorno propio para el desarrollo en PHP, poseer refactorizaciones más potentes, atajos de teclado propios de un entorno de desarrollo, entre otras.

\subsection{PHPUnit}

PHPUnit es un framework para el desarrollo de pruebas unitarias en PHP. \cite{wiki:phpunit}. Permite construir y ejecutar pruebas unitarias o de cobertura código PHP. 

Se ha decidido utilizar este framework de pruebas al venir integrado en la propia distribución de XAMPP.

\subsection{SonarQube}

SonarQube es una plataforma libre que permite realizar una  integración continua de nuestro desarrollo software. Informa sobre la duplicidad de código, errores en el no cumplimiento de estándares de configuración, errores en el diseño del software, pruebas unitarias, etc. \cite{wiki:sonarqube}

Se ha decidido utilizar esta herramienta al ser propuesta por el propio tribunal.

\subsection{Advanced REST Client}

Advanced REST Client software que permite realizar peticiones REST sobre nuestro propio servidor. \cite{wiki:aresrclient}

Se ha decidido utilizar esta herramienta frente a otras alternativas como \emph{Postman} \cite{wiki:postman}, al ser enfocada únicamente para un servidor REST. Esta herramienta ha sido utilizada para depurar de manera manual cada una de las peticiones realizadas al APIREST.

\subsection{StarUML}

StarUML es un software de diseño UML que permite la construcción de diagramas de clases, secuencias, paquetes, entre otras. \cite{wiki:staruml}

Se ha decidido utilizar esta herramienta frente a otras como \emph{Astah*} \cite{wiki:astah} al ser utilizado anteriormente.

\subsection{Librerías}

A continuación, se incluye una pequeña explicación sobre las librerías utilizadas para el desarrollo del proyecto.

\subsubsection{Consumir peticiones}

Durante el desarrollo de la aplicación se han barajado ambas librerías para en consumo de las peticiones al propio API y al webservice de Moodle. Finalmente, se ha decido optar por la librería Retrofit al ser más fácil de utilizar, más intuitiva a la hora de crear y configurar las distintas peticiones, más flexible y suele tener mejores resultados que Volley

\begin{itemize}

	\item \textbf{Retrofit:} Retrofit proporciona un framework para poder interactuar con APIs y enviar peticiones HTTP de forma fiable desde aplicaciones Android. Para ello proporciona un cliente HTTP encargado de interactuar con APIs y de gestionar las peticiones de manera transparente.
Esta librería es utilizada para el tratamiento de las peticiones desde la aplicación android. Además, permite consumir peticiones de manera síncrona y asíncrona. \cite{wiki:retrofit}
	
	\item \textbf{Volley:} Al igual que la librería anterior, Volley es una librería desarrollada por Google para enviar peticiones HTTP desde aplicaciones Android. \cite{wiki:volley}
	
	\item \textbf{Gson:} Esta librería proporcionada por Google, permite trabajar con JSON a la hora de serializar y deserializar los objetos. Es utilizada para convertir las respuestas JSON de las APIs en objetos Java de manera muy sencilla, eliminado toda esta carga de conversión al programador. Además, también permite el paso contrario a la hora de enviar objetos a las APIs, al encargarse de transformar estos objetos en respuestas en formato JSON. \cite{wiki:gson}

\end{itemize}

\subsubsection{JUnit}

JUnit es un framework para realizar pruebas unitarias sobre aplicaciones Java. \cite{wiki:junit}

Se ha utilizado esta herramienta para realizar las pruebas unitarias sobre la aplicación Android.

\subsubsection{Espresso}

Espresso es el propio framework de Android para la realización de pruebas de interfaz de usuario de manera automatizada. \cite{wiki:espresso}

Se ha decidido utilizar esta herramienta al ser la recomendada para realizar la automatización de pruebas de UI.

\section{Técnicas}

\subsection{SCRUM}\label{scrum} 

SCRUM es un modelo de referencia que define un conjunto de práctica y roles, que pueden utilizarse como punto de partida para llevar a cabo la definición de como se elaborará el proceso de desarrollo del proyecto \cite{wiki:scrum}. Cabe destacar que este modelo se encuentra dentro de los marcos de las metodologías ágiles y que es de los más utilizados actualmente.
Los roles principales son:

\begin{itemize}

	\item \textbf{Scrum Master:} se encarga de gestionar los cambios y procura facilitar la aplicación de esta metodología.
	
	\item \textbf{Product Owner:} en este grupo se encontrarán el personal interno o externo que representa al cliente. Este grupo se encargará de que el trabajo se realice de forma acorde con las necesidades y peticiones del cliente.
	
	\item \textbf{Team:} representa al equipo de desarrollo encargados de ejecutar el desarrollo y de entregar el producto deseado.
	
	\item \textbf{Product Owner:} en este grupo se encontrarán el personal interno o externo que representa al cliente. Este grupo se encargará de que el trabajo se realice de forma acorde con las necesidades y peticiones del cliente.	

\end{itemize}
\cite{wiki:scrum}
 
Para realizar el desarrollo del producto se deben de definir un Product Backlog, el cual contendrá todas las historias de usuario a realizar. Dichas historias de usuario se asignarán a los distintos Sprints o iteraciones que se realicen a lo largo del desarrollo del producto hasta finalizar con todas las historias de usuario. 










