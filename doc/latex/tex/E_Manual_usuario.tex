\apendice{Documentación de usuario}

\section{Introducción}

Este manual se presentan y explican los requisitos necesarios para utilizar la aplicación, como instalar en el un dispositivo físico y otras pautas de configuración para su correcto funcionamiento.

\section{Requisitos de usuarios}

Para poder utilizar la aplicación es necesario cumplir una serie de requisitos mínimos:

\begin{itemize}

	\item Tener instalado  las herramientas explicadas en \ref{subsec:RequisitosMinimos}.
	\item Configurar las herramientas como se encuentran explicadas en \ref{subsubsec:InstHerramientas}.

\end{itemize}

\section{Instalación}

\subsection{Configuración}\label{subsec:ConfiguracionMaquina}

La aplicación ha sido desarrollada para dispositivos Android 4.1 (Jelly Bean) o superiores. Antes de iniciar la aplicación es necesario realizar unos pequeños cambios en la configuración.

\begin{enumerate}
	
	\item Modificar la dirección ip predeterminada del archivo de configuración de Moodle. Para ello tendremos que:
	\begin{enumerate}
		\item Situarnos dentro de la propia carpeta de la herramienta: 

D:\textbackslash{}xampp\textbackslash{}htdocs\textbackslash{}moodle
		\item Una vez situados en dicha ruta tendremos que modificar la dirección del host: \emph{localhost} por la dirección ip de la propia máquina del archivo config.php como podemos ver en la siguiente ilustración.
		\imagen{archivoConfig}{Archivo config.php de Moodle.}
	\end{enumerate}		
	
	\item Al haber modificado el host de Moodle tendremos que modificar la dirección de inicio de todas nuestras herramientas externas QuickTest por la dirección ip de la maquina. Esto es porque tenemos QuickTest y Moodle en la misma máquina, sino no seria necesario. \imagenAncho{modUrlInicio}{Modificando la URL de inicio de las herramientas externas.}{1}
	
	\item A continuación, tendremos que abrir nuestro proyecto en Android Studio y modificar los siguientes archivos dentro del paquete api. \imagenAncho{ModificarapiAndroid}{Archivos a modificar dentro del paquete api.}{0.8}
	
	\begin{enumerate}
		
		\item \emph{APIMoodle:} tendremos que modificar la variable \emph{BASE} por la dirección de la máquina y las variables \emph{USERNAME, PASSWORD} por el nombre y contraseña del usuario encargado de manejar el web services de Moodle, ya que el es el único con privilegios para realizar las peticiones de información. \imagen{ModAPIMoodle}{Modificamos las variables de la clase APIMoodle}
		
		\item \emph{APIRest:} tendremos que modificar la variable \emph{BASE} por la dirección de la máquina. \imagen{ModAPIRest}{Modificamos la variable de la clase APIRest.}
		
	\end{enumerate}
	 
\end{enumerate}

Una vez realizada esta breve modificación en la configuración, la aplicación Android se encuentra completamente operativa para su ejecución desde el propio emulador de Android Studio o desde nuestro propio dispositivo Android.

\subsection{Instalación en un dispositivo Android físico}

Para llevar cabo la instalación de la aplicación en nuestro propio dispositivo Android es necesario:

\begin{itemize}
	\item Haber realizado la breve configuración previa del punto anterior. \ref{subsec:ConfiguracionMaquina}.
	
	\item Conectar nuestro dispositivo al ordenador.
	
	\item Nuestro dispositivo pertenezca a la misma red que nuestro equipo.
\end{itemize}

Una vez realizado estos pasos tendremos que:

\begin{enumerate}
	\item Arrancar el proyecto en Android Studio.
	\item Ejecutar la aplicación y seleccionar como dispositivo de destino, nuestro móvil.\imagen{InstApp}{Instalación de la aplicación en un dispositivo físico.}
\end{enumerate}

Una vez realizados estos pasos podremos utilizar nuestra aplicación desde nuestro propio dispositivo Android.

\section{Manual del usuario}

En esta sección se describen el funcionamiento de las funcionalidades desarrolladas de la aplicación.

\subsection{Iniciar Sesión}

Al iniciar la aplicación por primera vez nos aparecerá la ventana de \emph{inicio de sesión.} Una vez allí tendremos que:

\begin{enumerate}
	\item Introducir nuestro nombre de usuario y contraseña de Moodle.
	\item Si se quiere recordar estas credenciales para que en futuros acceso esta ventana no se active, bastará con activar la opción \emph{Recordar campos.}
	\item Click en \emph{Iniciar Sesión.}
\end{enumerate}

Una vez iniciada sesión la aplicación nos mostrará los cursos en los que el usuario se encuentra matriculado independientemente de su rol en los mismos.

\imagen{IniciarSesion}{Iniciar sesión en la aplicación.}



\subsection{Obtener la calificación de los alumnos}

Para poder comprobar la calificación de los alumnos de un profesor simplemente tendremos que:

\begin{enumerate}
	\item Entrar en el curso.
	\item Seleccionar el cuestionario a comprobar.
	\item Comprobar la calificación de cada alumno.
\end{enumerate}

\imagen{ObtenerCalAlumnos}{Obtener la calificación de los alumnos matriculados en un curso.}

Como podemos observar en la anterior ilustración, se utiliza el siguiente criterio para determinar que cuestionarios han sido resueltos de la aplicación web, desde la aplicación Android o aun no han sido resueltos.

\begin{itemize}
	\item \emph{Icono de Moodle:} cuestionario resuelto desde la aplicación web.
	
	\item \emph{Icono de información:} cuestionario sin resolver.
	
	\item \emph{Calificación:} cuestionario resuelto desde la aplicación Android.
\end{itemize}

\subsection{Mostrar cuestionario}

Para obtener el cuestionario ya completado por los alumnos tendremos que:

\begin{enumerate}
	\item Entrar en el curso.
	\item Seleccionar el cuestionario a comprobar.
	\item Seleccionar, \emph{Ver cuestionario}.
\end{enumerate}

\imagen{VerCuestionario}{Obtenemos el cuestionario al que se han enfrentado los alumnos.}

\subsection{Resolver un cuestionario}

Para resolver un cuestionario tendremos que:
\begin{enumerate}
	\item Entrar en el curso.
	\item Posicionarnos en \emph{Mis cuestionarios sin resolver} en el \emph{Burger Menu}.
	\item Click largo sobre el cuestionario a resolver y seleccionamos \emph{Resolver}.
	\item Una vez resueltas las preguntas, click en \emph{Enviar y finalizar cuestionario} y en la ventana emergente seleccionar  \emph{Enviar.}
\end{enumerate}

\imagen{ResolverCuestionario}{Resolvemos un cuestionario.}


\subsection{Revisar un cuestionario}

Para revisar un cuestionario resuelto tendremos que:
\begin{enumerate}
	\item Entrar en el curso.
	\item Posicionarnos en \emph{Mis cuestionarios resueltos} en el \emph{Burger Menu.}
	\item Click largo sobre el cuestionario a revisar y seleccionamos \emph{Revisar.}
	\item Si deseamos obtener información adicional sobre como ha sido resuelto el cuestionario, hacer click en \emph{Ver resultados}
\end{enumerate}

\imagen{RevisarCuestionario}{Revisar un cuestionario.}

\subsection{Cerrar sesión}

Para cerrar sesión u olvidar y cerrar sesión tendremos que:

\begin{itemize}
	\item \emph{Rol alumno:}
	\begin{enumerate}
		\item Entrar en el curso.
		\item Abrir en \emph{Burger Menu} y seleccionar \emph{Cerrar sesión u Olvidar y cerrar sesión.}
	\end{enumerate}
	\item \emph{Rol Profesor:}
		\begin{enumerate}
			\item Entrar en el curso.
			\item Abrir el botón de ajustes y seleccionar \emph{Cerrar sesión u Olvidar y cerrar sesión.}
		\end{enumerate}
\end{itemize}

\imagen{CerrarSesion}{Cerrar sesión desde ambos roles.}
