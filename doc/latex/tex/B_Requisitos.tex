\apendice{Especificación de Requisitos}

\section{Introducción}

Como ya se ha comentado anteriormente, se va utilizar la metodología ágil SCRUM. Es por esto que no vamos a tratar las tareas a realizar como requisitos, sino como historias de usuario. Todas estas historias de usuario formarán parte del Product Backlog y serán asignadas a determinadas iteraciones durante el desarrollo del proyecto.
A continuación, se incluye un listado sobre las historias de usuarios y sus correspondientes diagramas de casos de uso.


\section{Product Backlog}

En esta sección se incluyen todas aquellas historias de usuarios necesarias para los alumnos y los profesores.

\subsection{Historias de usuario de ambos roles}

\subsubsection{Historia de usuario 1: Iniciar sesión}\label{subsec:HU1}

\tablaSmall{Historia de usuario 1}{l l}{coste-1}
{ \multicolumn{1}{l}{HU1}\\}{ 
Título & Como un usuario autenticado, podrá iniciar sesión. \\
Rol & Alumno o profesor autenticado. \\
Descripción & El usuario introducirá sus credenciales y se le mostrará \\\
& aquellos cursos en los que se encuentre matriculado. \\
Precondiciones & - El usuario debe estar previamente registrado. \\\
			   & - Haber iniciado sesión. \\
}

\subsubsection{Historia de usuario 2: Cerrar sesión}\label{subsec:HU2}

\tablaSmall{Historia de usuario 2}{l l}{coste-1}
{ \multicolumn{1}{l}{HU2}\\}{ 
Título & Como un usuario autenticado, podrá cerrar sesión.\\
Rol & Alumno o profesor autenticado. \\
Descripción & Siempre y cuando el usuario se encuentre logeado, podrá\\\
& cerrar sesión.\\
Precondiciones & - El usuario debe estar previamente registrado.\\\
& - Haber iniciado sesión. \\
}

\subsubsection{Historia de usuario 3: Recordar campos}\label{subsec:HU3}

\tablaSmall{Historia de usuario 3}{l l}{coste-1}
{ \multicolumn{1}{l}{HU3}\\}{ 
Título & Como un usuario autenticado, podrá recordar los campos.\\
Rol & Alumno o profesor autenticado. \\
Descripción & El sistema permitirá recordar los campos con los \\\
& que el usuario inicia sesión.\\
Precondiciones & - El usuario debe estar previamente registrado.\\\
& - Haber iniciado sesión. \\
}

\subsubsection{Historia de usuario 4: Cursos}\label{subsec:HU4}

\tablaSmall{Historia de usuario 4}{l l}{coste-1}
{ \multicolumn{1}{l}{HU4}\\}{ 
Título & Como un usuario autenticado, podrá ver sus cursos.\\
Rol & Alumno o profesor autenticado. \\
Descripción & El sistema mostrará los cursos en los que se encuentra \\\ & matriculado.\\
Precondiciones & - El usuario debe estar previamente registrado.\\\
& - Haber iniciado sesión. \\
}

\subsection{Historias de usuario de los alumnos}

\subsubsection{Historia de usuario 5: Cuestionarios}\label{subsec:HU5}

\tablaSmall{Historia de usuario 5}{l l}{coste-1}
{ \multicolumn{1}{l}{HU5}\\}{ 
Título & Como un usuario autenticado, el sistema notificará sobre\\\
& los cuestionarios a resolver. \\
Rol & Alumno autenticado. \\
Descripción & Una vez el usuario haya iniciado sesión obtendrá todos\\\  & aquellos cuestionarios en los que se encuentre asignado. \\
Precondiciones & - El usuario debe estar previamente registrado.\\\
& - Haber iniciado sesión. \\\
& - Existan cuestionarios en aquellos cursos en los que\\\
& se encuentra matriculado el alumno. \\
}

\subsubsection{Historia de usuario 6: Resolver cuestionario}\label{subsec:HU6}

\tablaSmall{Historia de usuario 6}{l l}{coste-1}
{ \multicolumn{1}{l}{HU6}\\}{ 
Título & Como un usuario autenticado, podrá resolver un \\\
& cuestionario. \\
Rol & Alumno autenticado. \\
Descripción & Una vez el usuario haya iniciado sesión podrá resolver \\\ & aquellos cuestionarios en los que se encuentren asignado. \\
Precondiciones & - El usuario debe estar previamente registrado.\\\
& - Haber iniciado sesión. \\\
& - Existan cuestionarios en aquellos cursos en los que se\\\
& encuentra matriculado el alumno. \\
}

\subsubsection{Historia de usuario 7: Finalizar cuestionario}\label{subsec:HU7}

\tablaSmall{Historia de usuario 7}{l l}{coste-1}
{ \multicolumn{1}{l}{HU7}\\}{ 
Título & Como un usuario autenticado, podrá finalizar un \\\ & cuestionario. \\
Rol & Alumno autenticado. \\
Descripción & El usuario podrá finalizar aquel cuestionario que\\\
& este resolviendo. \\
Precondiciones & - El usuario debe estar previamente registrado.\\\
& - Haber iniciado sesión. \\\
& - Existan cuestionarios en aquellos cursos en los que se \\\
& encuentra matriculado el alumno. \\\
& - Haber iniciado la resolución de un cuestionario. \\
}

\subsubsection{Historia de usuario 8: Calificación obtenida}\label{subsec:HU8}

\tablaSmall{Historia de usuario 8}{l l}{coste-1}
{ \multicolumn{1}{l}{HU8}\\}{ 
Título & Como un usuario autenticado, el sistema notificará \\\
& sobre la calificación obtenida. \\
Rol & Alumno autenticado. \\
Descripción & Una vez el usuario finalice el cuestionario, el sistema \\\
& proporcionará una retroalimentación sobre el cuestionario \\\
& resuelto.\\
Precondiciones & - El usuario debe estar previamente registrado.\\\
& - Haber iniciado sesión. \\\
& - Existan cuestionarios en aquellos cursos en los que se\\\
&  encuentra  matriculado el alumno. \\\
& - Haber iniciado la resolución de un cuestionario. \\\
& - Haber finalizado un cuestionario. \\
}

\subsubsection{Historia de usuario 9: Revisar cuestionario resuelto}\label{subsec:HU9}

\tablaSmall{Historia de usuario 9}{l l}{coste-1}
{ \multicolumn{1}{l}{HU9}\\}{ 
Título & Como un usuario autenticado, podrá revisar un \\\
& cuestionario resuelto.\\\
Rol & Alumno autenticado. \\
Descripción & El usuario podrá revisar la retroalimentación de un  \\\ & cuestionario que ya esté finalizado. \\
Precondiciones & - El usuario debe estar previamente registrado.\\\
& - Haber iniciado sesión. \\\
& - Existan cuestionarios en aquellos cursos en los que se\\\
&  encuentra matriculado el alumno. \\\
& - Haber iniciado la resolución de un cuestionario. \\\
& - Haber finalizado un cuestionario. \\
}

\subsection{Historias de usuario de los profesores}

\subsubsection{Historia de usuario 10: Cuestionarios}\label{subsec:HU10}

\tablaSmall{Historia de usuario 10}{l l}{coste-1}
{ \multicolumn{1}{l}{HU10}\\}{ 
Título & Como un usuario autenticado, podrá ver los cuestionarios \\\ & de un curso a los que se enfrentan sus alumnos.\\
Rol & Profesor autenticado. \\
Descripción & El sistema mostrará al profesor los cuestionarios a los \\\
& que se enfrentan sus alumnos.\\
Precondiciones & - El usuario debe estar previamente registrado.\\\
& - Haber iniciado sesión. \\
}

\subsubsection{Historia de usuario 11: Calificaciones}\label{subsec:HU11}

\tablaSmall{Historia de usuario 11}{l l}{coste-1}
{ \multicolumn{1}{l}{HU11}\\}{ 
Título & Como un usuario autenticado, podrá ver las calificaciones \\\ & de los alumnos en un cuestionario.\\
Rol & Profesor autenticado. \\
Descripción & El sistema mostrará al profesor la calificación de cada \\\ & alumno en cada cuestionario.\\
Precondiciones & - El usuario debe estar previamente registrado.\\\
& - Haber iniciado sesión. \\
}

\subsubsection{Historia de usuario 12: Ver cuestionarios}\label{subsec:HU12}

\tablaSmall{Historia de usuario 12}{l l}{coste-1}
{ \multicolumn{1}{l}{HU12}\\}{ 
Título & Como un usuario autenticado, podrá ver los cuestionarios.\\
Rol & Profesor autenticado. \\
Descripción & El sistema mostrará al profesor los cuestionarios a los que se \\\ & enfrentan sus alumnos.\\
Precondiciones & - El usuario debe estar previamente registrado.\\\
& - Haber iniciado sesión. \\
}

\section{Diagrama de casos de uso}

En la siguiente ilustración podemos ver el diagrama de los casos de uso de nuestro sistema:

\imagen{DiagramaCasosDeUso}{Diagrama de casos de uso del alumno.}

\imagen{DiagramaCasosDeUsoProfesor}{Diagrama de casos de uso del profesor.}

\subsection{Caso de uso 1: Iniciar sesión}

\tablaSmall{Caso de uso 1: Iniciar sesión}{l l}{coste-1}
{ \multicolumn{1}{l}{CU1}\\}{ 
Título & Iniciar sesión.\\
Descripción & El sistema permitirá al usuario autenticado iniciar sesión para \\\ & acceder al sistema.\\
Secuencia & 1. El usuario introducirá los campos de nombre y contraseña. \\\
& 2. El sistema comprobará si los campos son correctos: \\\
& \hspace{0.25cm} 2.1. Si son correctos, el usuario pasa a estar logeado teniendo  \\\ & \hspace{0.25cm} pleno acceso a las funcionalidades del sistema. \\\
& \hspace{0.25cm} 2.2. Si no son correctos, el sistema notificará que los campos \\\ & son incorrectos. \\
Precondiciones & - El usuario debe estar previamente registrado en Moodle. \\
Comentarios & Este caso de uso, CU1, corresponde a la HU1. \ref{subsec:HU1} \\	
}

\subsection{Caso de uso 2: Olvidar campos}

\tablaSmall{Caso de uso 2: Olvidar campos}{l l}{coste-1}
{ \multicolumn{1}{l}{CU2}\\}{ 
Título & Recordar campos.\\
Descripción & El sistema permitirá al usuario olvidar los campos con las que \\\ & el usuario inicia sesión.\\
Secuencia & 1. El usuario seleccionara en el sistema cerrar sesión. \\\
& 2. El sistema olvidará las credenciales del usuario.\\
Precondiciones & - El usuario debe estar previamente registrado en Moodle.\\\
& - El usuario debe haber iniciado sesión y haber activado la 
\\\ & opción  de recordar campos.\\
Comentarios & Este caso de uso, CU2, corresponde a la historia de usuario 2. \\\ & \ref{subsec:HU2} \\
}

\subsection{Caso de uso 3: Recordar campos}

\tablaSmall{Caso de uso 3: Recordar campos}{l l}{coste-1}
{ \multicolumn{1}{l}{CU3}\\}{ 
Título & Olvidar campos.\\
Descripción & El sistema permitirá al usuario recordar los campos con las que \\\ & el usuario inicia sesión.\\
Secuencia & 1. El sistema recordará las credenciales del usuario. \\
Precondiciones & - El usuario debe estar previamente registrado en Moodle.\\
Comentarios & Este caso de uso, CU3, corresponde a la historia de usuario 3. \\\ & \ref{subsec:HU3}\\
}

\subsection{Caso de uso 4: Mostrar cursos}

\tablaSmall{Caso de uso 4: Mostrar cuestionarios}{l l}{coste-1}
{ \multicolumn{1}{l}{CU4}\\}{ 
Título & Mostrar cursos.\\
Descripción & El sistema permitirá al usuario ver los cursos en \\\
& los que se encuentra matriculado.\\
Secuencia & 1. El sistema mostrará los cursos. \\
Precondiciones & - El usuario debe estar previamente registrado en Moodle. \\\ & - El usuario debe estar logeado en la aplicación para poder  \\\
& acceder a esta funcionalidad. \\\
& - Deben existir cuestionarios de QuickTest en cursos en los que  \\\ & el usuario se encuentre matriculado.\\
Comentarios & Este caso de uso, CU4, corresponde a la historia de usuario 4. \\\ & \ref{subsec:HU4} \\
}

En esta ilustración podemos ver el diagrama de casos de uso de los casos de los anteriores casos de uso expuestos.
\imagen{DiagramaCasosDeUsoCU1234}{Diagrama de casos de uso de CU1, CU2, CU3 y CU4.}


\subsection{Caso de uso 5: Mostrar cuestionarios}

\tablaSmall{Caso de uso 5: Mostrar cuestionarios}{l l}{coste-1}
{ \multicolumn{1}{l}{CU5}\\}{ 
Título & Mostrar cuestionarios.\\
Descripción & El sistema permitirá al usuario ver los cuestionarios de  \\\
& un curso.\\
Secuencia & 1. El sistema mostrará los cuestionarios. \\
Precondiciones & - El usuario debe estar previamente registrado en Moodle. \\\ & - El usuario debe estar logeado en la aplicación para poder  \\\
& acceder a esta funcionalidad. \\\
& - Deben existir cuestionarios de QuickTest en cursos en los que  \\\ & el usuario se encuentre matriculado.\\
Comentarios & Este caso de uso, CU5, corresponde a la historia de usuario 5. \\\ & \ref{subsec:HU5} \\
}

\subsection{Caso de uso 6: Resolver cuestionario}

\tablaSmall{Caso de uso 6: Resolver cuestionario}{l l}{coste-1}
{ \multicolumn{1}{l}{CU6}\\}{ 
Título & Resolver cuestionario.\\
Descripción & El sistema permitirá al usuario resolver un cuestionario. \\
Secuencia & 1. El usuario seleccionará un cuestionario a resolver y comenzará \\\ & \hspace{0.25cm} a resolverlo. \\\
&  2. Una vez finalizado el cuestionario, el sistema enviará el \\\
& \hspace{0.25cm} cuestionario resuelto HU7. \ref{subsec:HU7} \\\
& 3. El sistema mostrará la retroalimentación del cuestionario \\\ & \hspace{0.25cm} resuelto HU8, HU9. \ref{subsec:HU8}\\
Precondiciones & - El usuario debe estar previamente registrado en Moodle.\\\
& - El usuario debe estar logeado en la aplicación para poder 
\\\ & acceder a esta funcionalidad. \\\
& - Deben existir cuestionarios de QuickTest en cursos en los que \\\ & el usuario se encuentre matriculado y que hayan sido resueltos.\\
Comentarios & Este caso de uso, CU6, corresponde a la historia de usuario \\\ & 6, 7, 8, 9.  \ref{subsec:HU6}\\
}

\subsection{Caso de uso 7: Mostrar cuestionarios resueltos}

\tablaSmall{Caso de uso 7: Mostrar cuestionarios resueltos}{l l}{coste-1}
{ \multicolumn{1}{l}{CU7}\\}{ 
Título & Mostrar cuestionarios resueltos.\\
Descripción & El sistema permitirá al usuario ver los cuestionarios resueltos. \\
Secuencia & 1. El sistema mostrará los cuestionarios resueltos. \\\
& \hspace{0.25cm} 1.1. Si existe algún cuestionario resuelto el usuario   \\\ & \hspace{0.25cm} podrá seleccionar un cuestionario y obtener la  \\\ & \hspace{0.25cm} retroalimentación del mismo HU8, HU9. \\
Precondiciones & - El usuario debe estar previamente registrado en Moodle.\\\
& - El usuario debe estar logeado en la aplicación para poder \\\ & acceder a esta funcionalidad. \\\
& - Deben existir cuestionarios de QuickTest en cursos en los que   \\\ & el usuario se encuentre matriculado y que hayan sido resueltos.\\
Comentarios & Este caso de uso, CU7, corresponde a la historia de usuario 6. \\\ & \ref{subsec:HU7}\\
}

En esta ilustración podemos ver el diagrama de casos de uso de los casos de los anteriores casos de uso expuestos.
\imagen{DiagramaCasosDeUsoCU567}{Diagrama de casos de uso de CU5, CU6 y CU7.}

\subsection{Caso de uso 8: Mostrar calificación de los alumnos}

\tablaSmall{Caso de uso 8: Mostrar calificación de los alumnos}{l l}{coste-1}
{ \multicolumn{1}{l}{CU8}\\}{ 
Título & Mostrar calificación de los alumnos.\\
Descripción & El sistema permitirá al usuario ver la calificación de los   \\\
& alumnos en un cuestionario.\\
Secuencia & 1. El usuario seleccionará un cuestionario. H10 \ref{subsec:HU11}\\\
& 2. Obtendrá la calificación de cada alumno en dicho\\\ &  cuestionario. HU11  \ref{subsec:HU11}\\
Precondiciones & - El usuario debe estar previamente registrado en Moodle. \\\ & - El usuario debe estar logeado en la aplicación para poder  \\\
& acceder a esta funcionalidad. \\\
& - Deben existir cuestionarios de QuickTest en cursos en los que  \\\ & el usuario se encuentre matriculado.\\
Comentarios & Este caso de uso, CU8, corresponde a la historia de usuario. \\\ & 10 y 11 \ref{subsec:HU10} \\
}

\subsection{Caso de uso 9: Ver cuestionarios}

\tablaSmall{Caso de uso 9: Ver cuestionarios}{l l}{coste-1}
{ \multicolumn{1}{l}{CU9}\\}{ 
Título & Ver cuestionarios.\\
Descripción & El sistema permitirá al usuario ver los cuestionarios, \\\
& es decir, las preguntas y posibles respuestas de un cuestionario.\\
Secuencia & 1. El sistema mostrará los cuestionarios. \\\
& 2. El usuario seleccionara ver cuestionario. \\
Precondiciones & - El usuario debe estar previamente registrado en Moodle. \\\ & - El usuario debe estar logeado en la aplicación para poder  \\\
& acceder a esta funcionalidad. \\\
& - Deben existir cuestionarios de QuickTest en cursos en los que  \\\ & el usuario se encuentre matriculado.\\
Comentarios & Este caso de uso, CU9, corresponde a la historia de usuario 12. \\\ & \ref{subsec:HU12} \\
}

En esta ilustración podemos ver el diagrama de casos de uso de los casos de los anteriores casos de uso expuestos.
\imagen{DiagramaCasosDeUsoCU89}{Diagrama de casos de uso de CU8 y CU9.}





