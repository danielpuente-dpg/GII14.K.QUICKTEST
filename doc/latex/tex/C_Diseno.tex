\apendice{Especificación de diseño}

\section{Introducción}

A continuación, se incluye una explicación/descripción del diseño de la aplicación Android. Para ello, se dividirán los diagramas en dos enfoques: el APIREST utilizado para comunicarse con el proyecto de partida y la propia app Android que se comunicará con dicho API para la obtención y volcado de los datos.

\section{Diseño de datos}

En este apartado se incluirán los diagramas de clases y paquetes de ambos enfoques.

\subsection{Diagrama de clases}

En estos diagramas se ha decidido utilizar un criterio de color para diferenciar aquellas clases que forman parte del APIREST, en color azul y en amarillo aquellos que forman parte del proyecto de partida.

\subsubsection{APIREST}

Este paquete será el encargado de comunicarse con el Controlador del proyecto de partida. A continuación, se mostrarán de manera fraccionada cada una de las clases que lo forman y con qué clases del controlador del proyecto de partida se comunican.

\paragraph{Clase ControlAccesoProfesor}

Esta clase será la encargada de realizar el control de acceso por parte de los profesores a la aplicación web. Esta web provee a los profesores de una clave privada dentro de la aplicación para poder utilizarla en la sincronización de sus cuestionarios con Moodle. Esta clase permite registrar a un nuevo profesor o comprobar si puede iniciar sesión. Ambas lógicas se controlan mediante una petición post. Para llevar a cabo este cometido se comunican con el controlador llamando al método registrarNuevoUsuario o login respectivamente.

\begin{itemize}

	\item \textbf{Post:} esta petición será la encargada de manejar la lógica de toda la clase.
	
	\begin{list}{-}
		\item \emph{Registrarse en QuickTest:} para ello es necesario incluir dentro del propio cuerpo de la petición el email y contraseña del usuario. La petición es la siguiente: \url{http://localhost/_QuickTest_TFG/app/apiREST/controlAccesoProfesor/registro}.
				
		\item \emph{Iniciar sesión en QuickTest:} para ello es necesario incluir en el cuerpo de la petición el email y la constraseña de usuario. La petición es la siguiente: \url{http://localhost/_QuickTest_TFG/app/apiREST/controlAccesoProfesor/login}. 
	\end{list}
	
\end{itemize}

\imagen{ControlAccesoProfesor}{Clase ControlAccesoProfesor}

\paragraph{Clase GestionCuestionario}

La lógica de esta clase será toda aquella que se encuentre relacionada con la gestión sobre los Cuestionarios: \emph{insertar, obtener, duplicar o borrar}. Cabe destacar, que en el controlador del proyecto de partida se ha incluido un método para poder duplicar un cuestionario existente, de esta manera evitamos comunicarnos con el modelo. Todas estas acciones se realizan mediante diferentes peticiones:

\begin{itemize}

	\item \textbf{Get:} esta petición será la encargada de mostrar aquellos cuestionarios dado un identificador de asignatura. \url{http://localhost/_QuickTest_TFG/app/apiREST/gestionCuestionario/id}
	\item \textbf{Post:} esta petición será la encargada de manejar dos lógicas.
		\begin{list}{-}
			\item \emph{Duplicar un cuestionario:} para ello es necesario indicar el identificador del cuestionario. La petición es la siguiente: \url{http://localhost/_QuickTest_TFG/app/apiREST/gestionCuestionario/duplicar/id}.
			\item \emph{Insertar/Editar un cuestionario:} para ello es necesario incluir en el propio cuerpo de la petición si la petición es de insertar o de duplicar. La petición es la siguiente: \url{http://localhost/_QuickTest_TFG/app/apiREST/gestionCuestionario/insertar}.
		\end{list}
	
	\item \textbf{Delete:} esta petición se encarga de eliminar un determinado cuestionario dado un identificador. La petición es la siguiente\url{http://localhost/_QuickTest_TFG/app/apiREST/gestionCuestionario/id}.

\end{itemize}

\imagen{GestionCuestionario}{Clase GestionCuestionario}

Cabe destacar, que estas dos clases explicadas anteriormente no han sido utilizadas finalmente en la aplicación, al no ser necesarias para su correcto funcionamiento. Estas clases fueron desarrolladas durante las primeras fases del proyecto. Estas primeras fases, se dedicaron a la creación del APIREST que posteriormente se utilizaría en la aplicación Android.

\paragraph{Clase SolucionCuestionario}

Esta clase se encarga de la lógica a realizar durante la resolución de un cuestionario: \emph{finalizar o mostrar resultado}.
Todas estas acciones se realizan mediante las siguientes peticiones:

\begin{itemize}

	\item \textbf{Post:} esta petición será la encargada de manejar la lógica a seguir al finalizar un cuestionario. Para ello, obtendrá el contenido del cuestionario resuelto de la petición, almacenará las respuestas del alumno y la nota del mismo. \url{http://localhost/_QuickTest_TFG/app/apiREST/solucionCuestionario/finalizar}.
	
	\item \textbf{Get:} esta petición sera la encargada de obtener la calificación de un alumno que haya resuelto el cuestionario desde la aplicación Android. Para ello, será necesario incluir en la propia petición el identificador del alumno y del cuestionario respectivamente. \url{http://localhost/_QuickTest_TFG/app/apiREST/solucionCuestionario/obtenerNota/idAlumno/idCuestionario}
\end{itemize}

\imagen{SolucionCuestionario}{Clase SolucionCuestionario}

\paragraph{Clase ObtenerCuestionario}

Esta clase se encarga de la obtención de las preguntas y respuestas que forman un cuestionario, así como que preguntas tienen o no comodín y al finalizar un cuestionario, obtener información acerca de como ha sido resuelto el mismo. Destacar,  que en el controlador no existe ningún método que permita obtener un cuestionario por identificador y el estado del mismo. Es por esto que esta funcionalidad ha sido incluida dentro del propio API. Además, ha sido necesario incluir dentro de este la funcionalidad de obtener los comodines verdes, ámbar y la información sobre como ha sido resuelto el cuestionario, ya que los métodos que se encuentran en el controlador corrompen la respuesta al incluir volcados del contenido.

\begin{itemize}

	\item \textbf{Get:} esta petición será la encargada de manejar toda la lógica de la clase.
	\begin{list}{-}
		\item \emph{Obtener cuestionario:} para ello sera necesario incluir en la propia petición el identificador del cuestionario. La petición es la siguiente: \url{http://localhost/_QuickTest_TFG/app/apiREST/obtenerCuestionario/obtener/idCuestionario}.
		
		\item \emph{Obtener el estado de un cuestionario:} es decir, si el cuestionario se encuentra resuelto o no por dicho alumno. Para ello, será necesarios incluir en la propia petición el identificador del alumno y el identificador de cuestionario. La petición es la siguiente:   \url{http://localhost/_QuickTest_TFG/app/apiREST/obtenerCuestionario/estado/idAlumno/idCuestionario}.
		
		\item \emph{Obtener comodín verde}: es decir, permite obtener aquella preguntas que tienen comodín verde, dado un cuestionario. Para ello es necesario incluir en la propia petición el identificador del cuestionario. La petición es la siguiente:\url{http://localhost/_QuickTest_TFG/app/apiREST/obtenerCuestionario/obtenerComodin/verde/idCuestionario}.
		
		\item \emph{Obtener comodín ámbar}: es decir, permite obtener aquella preguntas que tienen comodín ámbar, dado un cuestionario. Para ello es necesario incluir en la propia petición el identificador del cuestionario. La petición es la siguiente:\url{http://localhost/_QuickTest_TFG/app/apiREST/obtenerCuestionario/obtenerComodin/ambar/idCuestionario}.
		
		\item \emph{Obtener información sobre un cuestionario resuelto}: nos permite obtener información sobre somo ha sido resuelto un cuestionario, es decir, en que grupo se encuentra, cuantos comodines a utilizado, etc. Para ello es necesario incluir en la propia petición el identificador de cuestionario y alumno. La petición es la siguiente: \url{http://localhost/_QuickTest_TFG/app/apiREST/obtenerCuestionario/feedback/idCuestionario/idAlumno}.
	
	\end{list}

\end{itemize}

\imagen{ObtenerCuestionario}{Clase ObtenerCuestionario}

\subsubsection{Android}
En la siguiente ilustración podemos ver el diagrama de clases completo de la aplicación Android. Para ello, se ha decidido utilizar el color verde para representar dichas clases.

\begin{landscape}
\imagenAncho{DiagramaClasesAndroid}{Diagrama de clases de la aplicación.}{1.5}
\end{landscape}



\section{Diagrama de paquetes}

Para estos diagramas se ha decido utilizar el mismo criterio de color empleando anteriormente en el diagrama de clases.

\subsection{APIREST}

Este diagrama muestra un desglose de todos los componentes que entran en contacto en el API, como ya se ha comentado en color amarillo corresponde a la lógica dada en el proyecto de partida y en azul el APIREST desarrollado para interactuar con el controlador. Destacar que en color azul representa la estructura que forma el BackEnd de la aplicación. Toda esta estructura se encuentra en  
\_QuickTest\_TFG\_.app.apiREST y esta formada por:

\begin{itemize}

	\item \textbf{modelos:} en este paquete se encuentran todas aquellas clases que se comunican con el controlador del proyecto de partida.
	\item \textbf{test\_apiRest:} en este paquete se encuentran las pruebas unitarias creadas para probar comprobar el correcto funcionamiento del APIREST.
	\item \textbf{utilidades:} en este paquete se encuentran ciertas clases y ficheros de utilidad.
	\item \textbf{vista:} en este paquete se encuentran aquellas clases encargadas de gestionar la lógica de la vista de los datos.

\end{itemize}

\imagen{DiagramPaqRest}{Diagrama de paquetes del APIREST}

\subsection{Aplicación Android}

En este diagrama se muestra el desglose de todos los componentes que forman la arquitectura del FrontEnd de la aplicación. Toda esta estructura se encuentra dentro de QuickTest\_Android.app.src y esta formada por:

\begin{itemize}

	\item \textbf{main.activities:} en este paquete se encuentran todas las actividades necesarias para la construcción de la aplicación.
	\item \textbf{main.adapters:} en este paquete se encuentran todas los adaptadores creados para modificar el layout de las diferentes actividades y fragmentos.
	\item \textbf{main.api:} en esta paquete se encuentran todas clases encargadas de gestionar la correcta conexión de la aplicación con el APIREST y con el web services de Moodle.
	\item \textbf{main.fragments.students:} en este paquete se encuentran todos los fragmentos utilizados para la construcción de la aplicación.
	\item \textbf{main.models:} en este paquete se encuentran todas las clases necesarias para el tratamiento de los datos proporcionados por las diferentes peticiones, tanto al APIREST como al web services de Moodle. Es decir, estas clases son las encargadas de realizar la conversión de los datos en formato JSON a objetos Java o viceversa.
	\item \textbf{main.request:} en este paquete se encuentran todas aquellas clases encargadas de realizar una petición síncrona al APIREST o al web service de Moodle.
	\item \textbf{main.utils:} en este paquete se encuentra una clases encargada de la gestión de las SharedPreferences utilizadas para el correcto funcionamiento del login.
	\item \textbf{test:} en este paquete se encuentran las pruebas unitarias sobre el paquete main.models.
	\item \textbf{androidTest:} en este paquete se encuentra las pruebas automatizadas de interfaz de usuario sobre la aplicación.

\end{itemize}

\imagen{DiagramPaqAndroid}{Diagrama de paquetes de la aplicación Android}

\section{Diseño procedimental}

\subsection{Diagrama de secuencias}

En esta sección se incluyen los diagramas de secuencias o de interacción, encargados de mostrar de una manera mas aproximada la interacción del sistema al realizar cada caso de uso.

\subsubsection{Diagrama de secuencias que implementa CU1: Iniciar sesión}

\imagen{DiagramaSecuenciaCU1}{Diagrama de secuencias caso de uso 1.}

\subsubsection{Diagrama de secuencias que implementa CU2: Olvidar campos}

\imagen{DiagramaSecuenciaCU2}{Diagrama de secuencias caso de uso 2.}

\subsubsection{Diagrama de secuencias que implementa CU3: Recordar campos}

\imagen{DiagramaSecuenciaCU3}{Diagrama de secuencias caso de uso 3.}
\subsubsection{Diagrama de secuencias que implementa CU4: Mostrar cursos}
\imagen{DiagramaSecuenciaCU4}{Diagrama de secuencias caso de uso 4.}

\subsubsection{Diagrama de secuencias que implementa CU5: Mostrar cuestionarios}

\imagen{DiagramaSecuenciaCU5-Teacher}{Diagrama de secuencias caso de uso 5 desde el rol de profesor.}


\imagen{DiagramaSecuenciaCU5-Student}{Diagrama de secuencias caso de uso 5 desde el rol de alumno.}

\subsubsection{Diagrama de secuencias que implementa CU6: Resolver cuestionario}

\imagen{DiagramaSecuenciaCU6}{Diagrama de secuencias caso de uso 6.}

\subsubsection{Diagrama de secuencias que implementa CU7: Mostrar cuestionarios resueltos}

\imagen{DiagramaSecuenciaCU7}{Diagrama de secuencias caso de uso 7.}

\subsubsection{Diagrama de secuencias que implementa CU8: Mostrar calificación de los alumnos}

\imagen{DiagramaSecuenciaCU8}{Diagrama de secuencias caso de uso 8.}
\subsubsection{Diagrama de secuencias que implementa CU9: Ver cuestionarios}
\imagen{DiagramaSecuenciaCU9}{Diagrama de secuencias caso de uso 9.}

\section{Diseño arquitectónico}

A continuación, se muestra el diagrama de despliegue de todo el proyecto, incluyendo la funcionalidad del proyecto de partida. Para ello se utiliza el mismo criterio de color anteriormente utilizado.

\subsubsection{Diagrama de despliegue}

\imagen{DiagramaDespliege}{Diagrama de despliegue del proyecto.}



