\capitulo{1}{Introducción}


Este trabajo de final de grado nace de la base de un proyecto anterior, debido a la necesidad de ampliar la funcionalidad de dicho proyecto para poder ser empleado también desde una aplicación móvil desarrollada en Android, que permita llevar a cabo las mismas funcionalidades y acceder a los mismos contenidos a disposición del alumno por medio del sistema E-Learning de la Universidad de Burgos.

\section{Contenido del proyecto}

\subsection{Memoria}
\begin{itemize}
	\item \textbf{Introducción:} descripción breve sobre el proyecto a desarrollar.
	\item \textbf{Objetivos del proyecto:} principales objetivos del proyecto que posteriormente derivarán el las historias de usuario.
	\item \textbf{Conceptos teóricos:} principales conceptos teóricos necesarios para la comprensión del proyecto y del producto creado.
	\item \textbf{Aspectos relevantes del desarrollo del proyecto:} principales aspectos del proyecto.
	\item \textbf{Trabajos relacionados:} principales proyectos relacionados con producto creado.
	\item \textbf{Conclusiones y Líneas de trabajo futuras:} resumen sobre los conocimientos adquiridos y aspectos de mejora posibles en un futuro.
	
\end{itemize}

\subsection{Anexos}

\begin{itemize}
	\item \textbf{Plan de Proyecto Software:} explicación sobre la planificación seguida durante el desarrollo del proyecto, junto con un estudio de viabilidad legal.
	\item \textbf{Especificación de Requisitos:} explicación de las historias de usuario, casos de uso, etc.
	\item \textbf{Especificación de diseño:} explicación y descripción del diseño de la aplicación.
	\item \textbf{Documentación técnica de programación:} explicación de los requisitos mínimos y configuración para el funcionamiento del proyecto.
	\item \textbf{Documentación de usuario:} explicación sobre el funcionamiento de la aplicación.
	
\end{itemize}

\subsection{Proyecto}

El proyecto se encuentra disponible en el siguiente enlace:

\url{https://github.com/danielpuente-dpg/GII14.K.QUICKTEST}

