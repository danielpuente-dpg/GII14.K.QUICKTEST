\capitulo{6}{Trabajos relacionados}

Durante el desarrollo del proyecto se analizó el comportamiento y diseño de ciertas aplicaciones. No obstante, cabe destacar las siguientes.

\section{Proyecto de partida}

Como se ha comentado anteriormente, este proyecto nace por la necesidad de ampliar la funcionalidad de la resolución de este tipo de cuestionarios desde nuestro propio dispositivo Android, ya que presenta cierta problemática en el renderizado de la aplicación en un dispositivo móvil y al utilizar una aplicación nativa su funcionamiento sera más natural y fluido.

Al tratarse de una ampliación de este proyecto tanto las ventajas como inconvenientes que presenta el proyecto de partida serán los mismo que la aplicación desarrollada.


\section{Cuestionarios de Moodle}

El propio entorno educativo de Moodle incorpora sus propios cuestionarios. 

\subsection{Ventajas}

A diferencia de los propios cuestionarios de Moodle la aplicación permite:

\begin{itemize}
	\item \textbf{Dinamismo:} la herramienta construida permite que los alumnos se enfrenten a los cuestionarios de una manera más divertida, debido al sistema de recompensa empleado: \emph{comodines y orden.}
	
	\item \textbf{Alumnos:} estos podrán autoevaluar su nivel de conocimiento de la asignatura.
	
	\item \textbf{Docentes:} estos podrán evaluar en tiempo real el conocimiento de la clase de lo que acaba de explicar. De esta manera, podrá aclarar ciertos conceptos lo antes posible.
	
	\item \textbf{Comodines y orden:} los alumnos se enfrentarán a estos cuestionarios no solamente con la motivación de superarlo, sino que, \emph{si son los primeros obtendrán una recompensa}, \emph{si son de los últimos en responder tendrán a su disposición un mayor número de comodines}, es decir \emph{fomentará la competición entre los alumnos}, \emph{ayudará a que los propios alumnos se relacionen fruto de la competición}, etc.
	
	\item \textbf{Móvil:} al permitir que los usuarios puedan resolver estos cuestionarios desde su dispositivo Android, estos cuestionarios podrán realizarse como una comprobación al finalizar cada clase de lo aprendido en la misma, al necesitar únicamente de sus teléfonos móviles.
	
	
\end{itemize}

\subsection{Desventajas}

A continuación, se citan algunas de las principales desventajas que presenta  el producto creado frente a los cuestionarios nativos de Moodle.

\begin{itemize}
	\item \textbf{Multirespuesta:} permite que en el número de respuestas de una pregunta sea mayor que uno.
	
	\item \textbf{Categorías:} permite el almacenamiento de las preguntas por categorías.
	
	\item \textbf{Reintentos:} el profesor puede configurar el número de intentos de un mismo cuestionario.
	
	\item \textbf{Tiempo:} permite incluir un tiempo en el que el cuestionario tiene que resolverse.
	
	\item \textbf{Imagenes:} permite incluir en la propia pregunta una imagen.
	
	\item \textbf{Aleatoriedad:} permite que en un mismo cuestionario el número de preguntas varíen entre alumnos o intentos.
	
\end{itemize}


\section{Big Web Quiz}\cite{wiki:bigwebquiz}

Durante el desarrollo del proyecto se realizo una búsqueda de alguna herramienta de resolución de cuestionarios en dispositivos Android como complemento de Moodle pero, no se encontró ninguna al respecto. No obstante, se ha elegido esta herramienta: \emph{Big Web Quiz} al ser desarrollada por Google y al fomentar, al igual que la aplicación construida: \emph{\textbf{la competición entre los participantes.}}

\imagen{bigwebquiz}{Big Web Quiz en el Play Store.}

Ambas aplicaciones destacan por:
\begin{itemize}
	\item Fomentar la competición.
	\item Proporcionan una forma más dinámica de enfrentarse a un cuestionario.
	\item Permiten resolver los cuestionarios desde un dispositivo móvil Android.
	\item Fomentar la gamificacion durante la resolución de los cuestionaros, es decir aprender divirtiéndose.
\end{itemize}
	

\subsection{Ventajas}

\begin{itemize}
	\item \textbf{Moodle:} la aplicación desarrollada es compatible con LTI y por tanto con Moodle. Sin embargo, Big Web Quiz no. \ref{subsec:LTI}.
	
	\item \textbf{Docentes:} la aplicación desarrollada permite a los docentes obtener la calificación de sus alumnos.
	
	\item Para su funcionamiento no es necesario un Chromecast.
	
	\item Big Web Quiz solamente permite enfrentarse a un cuestionario desde un dispositivo móvil.
	
	\item La aplicación desarrollada permite que los alumnos puedan obtener información sobre como han resuelto un cuestionario y por medio de los comodines, relacionar su conocimiento con el del resto de la clase.
	
	\item Permite a los docentes crear sus propios cuestionarios.
\end{itemize}

\subsection{Desventajas}

A diferencia de la aplicación desarrollada, Big Web Quiz proporciona:
	\begin{itemize}
	\item Permite varias formas de respuesta: multirespuesta, preguntas con imágenes, proporciona gráficos, entre otras.
	
	\item Permite la selección de avatares al iniciar un cuestionario.
	
	\item Fomenta la competición de manera agrupada en varios cuestionarios.
	
	\end{itemize}
	





